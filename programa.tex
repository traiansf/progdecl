\documentclass[romanian]{article}
\usepackage{fullpage}
\usepackage{hyperref}
\usepackage[romanian]{babel}
\usepackage[utf8x]{inputenc}
\usepackage[T1]{fontenc}
\usepackage{ae,aecompl}
\usepackage[factor=700]{microtype}

\PrerenderUnicode{ȘșțȚăĂîÎâÂîÎ}
\usepackage{txfonts}

%\usepackage{tarticle}
%\lstset{language=Haskell}


\title{\vspace{-10ex}Programare declarativă}
\author{Programă}
\begin{document}
\maketitle
\thispagestyle{empty}

\vfill
\paragraph{Programare funcțională de ordinul I}
\begin{itemize}
\item Programare declarativă vs. programare imperativă: exemple și discuție.
\item Funcții de ordinul I; compunere; recursie.
\item Liste; pattern matching; operații de bază pe liste: transformare, selectare, agregare.
\end{itemize}
\paragraph{Programare funcțională cu funcții de ordin înalt}
\begin{itemize}
\item Funcții generice de transformare/selectare/agregare a elementelor dintr-o listă: Map, Filter, Fold
\item Currying; Operatori ca funcții și funcții ca operatori; aplicare parțială a funcțiilor/operatorilor
\item Funcții anonime și secțiuni
\end{itemize}
\paragraph{Tipuri de date algebrice}
\begin{itemize}
\item Tipuri de date algebrice: definire, exemple, utilizare.
\item Operații pe tipuri de date algebrice: parcurgere, transformare, agregare
\item Sintaxă abstractă; simplificarea și evaluarea expresiilor
\end{itemize}
\paragraph{Modularizare}
\begin{itemize}
\item Concepte OO realizate folosind modularizare: abstractizare, încapsulare, exportare
\end{itemize}
\paragraph{Clase de tipuri}
\begin{itemize}
\item Clase de tipuri.  Comparație cu supraîncarcarea din limbaje OO.
\item Metode implicite. Extinderi de clase; moștenire. 
\item Clase de tipuri standard: Eq, Ord, Show, tipuri aritmetice 
\item Generalizarea conceptelor algebrice folosind clase de tipuri: monoid; functor; produse si coproduse
\end{itemize} 
\paragraph{Efecte laterale: monade}
\begin{itemize}
\item Efecte laterale în limbaje pure. Dualitatea între intenție și acțiune
\item Monade: definiție și proprietăți
\item Exemple de monade: I/O; Liste/Nedeterminism; Analiză sintactică
\end{itemize}

\vfill
\paragraph{Bibliografie}
\begin{itemize}
\item  Learn You a Haskell for Great Good, de Miran Lipova\v{c}a

\url{https://www.haskell.org/tutorial/}
\item A Gentle Introduction To Haskell, version 98, de Paul Hudak, John Peterson, 
 Joseph Fasel
 
 \url{http://learnyouahaskell.com/}
 \item Haskell: The Craft of Functional Programming, Third Edition, de Simon Thompson
\end{itemize}
\end{document}