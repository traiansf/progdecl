\documentclass[xcolor=pdftex,romanian,colorlinks]{beamer}

\usepackage[export]{adjustbox}
\usepackage{../tslides}
\usepackage[all]{xy}
\usepackage{pgfplots}
\usepackage{flowchart}
\usetikzlibrary{arrows,positioning,calc}
\lstset{language=Haskell}
\lstset{escapeinside={(*@}{@*)}}

\AtBeginSection[]{
  \begin{frame}
  \vfill
  \centering
  \begin{beamercolorbox}[sep=8pt,center,shadow=true,rounded=true]{title}
    \usebeamerfont{title}\insertsectionhead\par%
  \end{beamercolorbox}
  \vfill
  \end{frame}
}


\title[PD---Intrare/Ieșire]{Programare declarativă}

\subtitle{Teoria categoriilor pentru programatori\thanks{bazata pe cartea \emph{\href{http://bartoszmilewski.com/2014/10/28/category-theory-for-programmers-the-preface/}{Category Theory for Programmers}} de \emph{Bartosz Milewski}}}

\begin{document}
\begin{frame}
  \titlepage
\end{frame}

\begin{frame}[fragile]{Category Theory for the Working C++ Programmer}{de Bartosz Milewski}
\href{https://www.youtube.com/watch?v=eCUfzvz7Z20}{http://www.youtube.com/watch?v=eCUfzvz7Z20}
\end{frame}

\section{Categorii}

\begin{frame}[fragile]{Categorii}
O categorie $\mathbb{C}$ este dată de:
\begin{itemize}
\item O clasă $|\mathbb{C}|$ a \structure{obiectelor} 
\item O clasă $\mathbb{C}$ a \structure{săgeților} / \structure{morfismelor}
\item Două funcții $\mathit{dom}, \mathit{cod} : \mathbb{C} \rightarrow |\mathbb{C}|$ care asociază fiecărei săgeți \structure{domeniul} și \structure{codomeniul} ei.  Pentru orice obiecte $A$ și $B$, \[\mathbb{C}(A,B) = \{f\in \mathbb{C}\mid \mathit{dom}(f) = A, \mathit{cod}(f) = B\}\] este \structure{mulțimea} săgeților  de la $A$ la $B$.

$f\in \mathbb{C}(A,B)$ poate fi scris ca $f : A \rightarrow B$ 
\item Pentru orice obiect $A$ o săgeată $\mathit{id}_A : A -> A$ numită \structure{identitatea} lui $A$
\item Pentru orice obiecte $A$, $B$, $C$, o operație de compunere a săgeților
$\circ : \mathbb{C}(B,C) \times \mathbb{C}(A,B) \rightarrow \mathbb{C}(A,C)$

\item Compunerea este asociativă și are element neutru $\textit{id}$
\end{itemize}
\end{frame}


\begin{frame}{Exemplu: Categoria $\mathrm{\mathbb{S}et}$}
\begin{itemize}
\item Obiecte: mulțimi
\item Săgeți: funcții
\item Identități:  Funcțiile identitate
\item Compunere: Compunerea funcțiilor
\end{itemize}

\begin{block}{Categorii asemănătoare lui $\mathrm{\mathbb{S}et}$}
\begin{itemize}
\item Categoria grupurilor cu morfisme de grupuri
\item Categoria algebrelor de o signatură dată
\end{itemize}
\end{block}
\end{frame}


\begin{frame}[fragile]{Exemplu: Categoria Hask}
\begin{itemize}
\item Obiectele: tipuri
\item Săgețiile: funcții între tipuri
\begin{asciihs}
f :: A -> B
\end{asciihs}
\item Identități: funcția polimorfică \lstinline$id$
\begin{asciihs}
Prelude> :t id
id :: a -> a
\end{asciihs}
\item Compunere: funcția polimorfică \lstinline$(.)$
\begin{asciihs}
Prelude> :t (.)
(.) :: (b -> c) -> (a -> b) -> a -> c
\end{asciihs}
\end{itemize}
\end{frame}


\begin{frame}{Categorii neasemănătoare lui $\mathrm{\mathbb{S}et}$}
{Categoria unui graf $(V,E)$}
\begin{itemize}
\item Obiecte: \only<2>{varfurile grafului $V$}
\item Săgeți: \only<2>{drumuri în grafuri}
\item Identități:  \only<2>{drumul vid de la un vârf la el însuși}
\item Compunere: \only<2>{alipirea drumurilor}
\end{itemize}
\end{frame}

\begin{frame}{Categorii neasemănătoare lui $\mathrm{\mathbb{S}et}$}
{Categoria unei relații parțiale de ordine $(S,\leq)$}
\begin{itemize}
\item Obiecte: \only<2>{elementele lui S}
\item Săgeți: \only<2>{perechile $(a,b)$ din $\leq$}
\item Identități:  \only<2>{garantate de reflexivitate}
\item Compunere: \only<2>{garantată de tranzitivitate}
\end{itemize}
\end{frame}

\begin{frame}{Categorii neasemănătoare lui $\mathrm{\mathbb{S}et}$}
{Categoria asociată unui monoid $(M,\cdot, e)$}
\begin{itemize}
\item Obiecte: \only<2->{un singur obiect, să-l numim $m$}
\item Săgeți: \only<2->{Elementele lui $M$}
\item Identități:  \only<2->{$e$}
\item Compunere: \only<2->{$\cdot$}
\end{itemize}

\onslide<3->
\vfill 
De fapt, merge și invers\dots

\begin{block}{Monoidul asociat unei categorii cu un singur obiect $|\mathbb{C}|=\{m\}$}
\begin{itemize}
\item $M = \mathbb{C} = \mathbb{C}(m,m)$
\item $\cdot = \circ$
\item $e = \mathit{id}_m$
\end{itemize}
\end{block}
\end{frame}

\begin{frame}{De ce categorii?}
\begin{block}{(Des)compunerea este esența programării}
\begin{itemize}
\item Am de rezolvat problema $P$
\item O descompun în subproblemele $P_1$,\dots $P_n$
\item Rezolv problemele $P_1$,\dots $P_n$ cu programele $p_1$,\dots $p_n$
\begin{itemize}
\item Eventual aplicând recursiv procedura de față
\end{itemize}
\item Compun rezolvările  $p_1$,\dots $p_n$ într-o rezolvare $p$ pentru problema inițială
\end{itemize}
\end{block}

\begin{block}{Categoriile rezolvă problema compunerii}
\begin{itemize}
\item Ne forțează să abstractizăm datele
\item Se poate acționa asupra datelor doar prin săgeți (metode?)
\item Forțează un stil de compunere independent de structura obiectelor
\end{itemize}
\end{block}
\end{frame}

\section{Monoizi}

\begin{frame}[fragile]{Monoizi}{Data.Monoid.Monoid}
\begin{asciihs}
class Monoid m where
    mempty  :: m
    mappend :: m -> m -> m
\end{asciihs}

\begin{block}{Monoidul listelor}
\vspace{-2ex}
\begin{asciihs}
instance Monoid [a] where
    mempty  = []
    mappend = (++)
\end{asciihs}
\end{block}
\end{frame}


\begin{frame}[fragile]{Monoide booleene}
\begin{block}{Monoidul conjunctiv\hfill Data.Monoid.All}
\vspace{-2ex}
\begin{asciihs}
newtype All = All { getAll :: Bool }

instance Monoid All where
        mempty = All True
        All x `mappend` All y = All (x && y)
\end{asciihs}
\end{block}
\begin{block}{Monoidul disjunctiv\hfill Data.Monoid.Any}
\vspace{-2ex}
\begin{asciihs}
newtype Any = Any { getAny :: Bool }

instance Monoid Any where
        mempty = Any False
        Any x `mappend` Any y = Any (x || y)
\end{asciihs}
\end{block}
\end{frame}


\begin{frame}[fragile]{Monoide numerice}
\begin{block}{Monoidul aditiv\hfill Data.Monoid.Sum}
\vspace{-2ex}
\begin{asciihs}
newtype Sum a = Sum { getSum :: a }

instance Num a => Monoid (Sum a) where
        mempty = Sum 0
        Sum x `mappend` Sum y = Sum (x + y)
\end{asciihs}
\end{block}
\begin{block}{Monoidul multiplicativ\hfill Data.Monoid.Product}
\vspace{-2ex}
\begin{asciihs}
newtype Product a = Product { getProduct :: a }

instance Num a => Monoid (Product a) where
        mempty = Product 1
        Product x `mappend` Product y = Product (x * y)
\end{asciihs}
\end{block}
\end{frame}


\begin{frame}[fragile]{La ce sunt buni Monoizii?}
{Data.Foldable.foldMap}
\begin{itemize}
\item Agregare într-un monoid:
\begin{asciihs}
    foldMap :: Monoid m => (a -> m) -> [a] -> m
    foldMap f = foldr (mappend . f) mempty
\end{asciihs}  
\end{itemize}
\begin{block}{Exemple}
\vspace{-2ex}
\begin{asciihs}
concat :: [[a]] -> [a]
concat = foldMap id
    
all :: [Bool] -> Bool         any :: [Bool] -> Bool
all = getAll . foldMap All    any = getAny . foldMap Any
    
sum :: Num a => [a] -> a      product :: Num a => [a] -> a
sum = getSum . foldMap Sum    product =
                                getProduct . foldMap Product
\end{asciihs}      
\end{block}
\end{frame}


\begin{frame}[fragile]{Monoidul endomorfismelor}
\begin{asciihs}
newtype Endo a = Endo { appEndo :: a -> a }

instance Monoid (Endo a) where
        mempty = Endo id
        Endo f `mappend` Endo g = Endo (f . g)
\end{asciihs}
\begin{itemize}
\item Mulțimea funcțiilor din $A$ în $A$ are structură de monoid
\end{itemize}
\end{frame}

\begin{frame}[fragile]{Relația între foldr și foldMap}
\begin{block}{foldMap în funcție de foldr}
\begin{asciihs}
    foldMap :: Monoid m => (a -> m) -> [a] -> m
    foldMap f = foldr (mappend . f) mempty
\end{asciihs}  
\end{block}

\begin{block}{foldr în funcție de foldMap}
\begin{asciihs}
    foldr :: (a -> b -> b) -> b -> [a] -> b
    foldr f z t = appEndo (foldMap (Endo . f) t) z
\end{asciihs}  

\structure{Observație:} \lstinline$f :: a -> (b -> b)$ translatează un element din a într-un endomorfism peste b
\end{block}
\end{frame}

\section{Obiecte speciale într-o categorie}

\begin{frame}[fragile]{Obiecte inițiale și finale}
\begin{block}{Obiect inițial}
Un obiect din care \structure{există} un \structure{singur} morfism către orice alt obiect
\begin{itemize}
\item mulțimea vidă în categoria Set
\item Algebra de termeni peste o signatură
\end{itemize}
\end{block}
\begin{block}{Obiect final}
Un obiect către care \structure{există} un \structure{singur} morfism din orice alt obiect
\begin{itemize}
\item O mulțime singleton în categoria Set
\item Algebra în care suportul fiecărui sort e singleton
\item Tipul \structure{()} în Hask

Pentru orice tip a, există o singură funcție de la a în ()
\begin{asciihs}
    unit :: a -> ()
    unit _ = ()
\end{asciihs}
\end{itemize}
\end{block}
\end{frame}

\section{Dualitate}

\begin{frame}{Duala unei categorii}
Dată fiind categoria $\mathbb{C}$, putem defini duala ei $\mathbb{C}^{op}$, prin simpla inversare a direcțiiei săgeților, astfel:
\begin{itemize}
\item $|\mathbb{C}^{op}|$ = $|\mathbb{C}|$
\hfill $\mathbb{C}^{op}$ = $\mathbb{C}$
\hfill $\mathit{dom}_{\mathbb{C}^{op}} = \mathit{codom}_{\mathbb{C}}$
\hfill $\mathit{codom}_{\mathbb{C}^{op}} = \mathit{dom}_{\mathbb{C}}$
\item $\circ_{\mathbb{C}^{op}} : \mathbb{C}^{op}(B,C) \times \mathbb{C}^{op}(A,B) \rightarrow \mathbb{C}^{op}(A,C)$ dată de:

pentru $f\in \mathbb{C}^{op}(B,C) = \mathbb{C}(C,B)$ și  $g\in \mathbb{C}^{op}(A,B) = \mathbb{C}(B,A)$, definim 
$$f \circ_{\mathbb{C}^{op}} g = g \circ_{\mathbb{C}} f \in  \mathbb{C}(C,A) = \mathbb{C}^{op}(A,C)$$
\item Identitățile rămân aceleași
\item Se verifică proprietățile de asociativitate
\end{itemize}

\begin{block}{Orice proprietate poate fi dualizată}
\begin{itemize}
\item Obiect final este obiect inițial în categoria duală
\item Obiect inițial este obiect final în categoria duală
\end{itemize}
\end{block}
\end{frame}

\section{Izomorfism}

\begin{frame}[fragile]{Izomorfism. Obiecte izomorfe}
Un morfism \lstinline$f :: A -> B$ se numește \structure{izomorfism}
dacă există un morfism \lstinline$g :: B -> A$ astfel încât
\vspace{-1ex}
\begin{asciihs}
f . g = id :: B -> B
g . f = id :: A -> A
\end{asciihs}
În acest caz, și g este izomorfism, iare obiectele A și B se numesc izomorfe


\begin{block}{Teoremă: Obiectul inițial e unic (modulo izomorfisme)}
\onslide<2>
Adică, oricare două obiecte inițiale într-o categorie sunt izomorfe

Fie A și B inițiale.  Atunci există (unice) \lstinline$f :: A -> B$ și \lstinline$g :: B -> A$. Avem:
\vspace{-1ex}
\begin{asciihs}
f . g :: B -> B                             id :: B -> B
g . f :: A -> A                             id :: A -> A
\end{asciihs}
Din unicitatea morfismelor de la A în A și B în B, rezultă
\vspace{-1ex}
\begin{asciihs}
f . g = id :: B -> B
g . f = id :: A -> A
\end{asciihs}
\end{block}
\structure{Din dualitate și obiectul final e unic (modulo izomorfime)}

\end{frame}

\section{Produse și coproduse}


\begin{frame}[fragile]{Produs}
Produsul a două obiecte a și b este obiectul c înzestrat cu două proiecții 
\lstinline$pa :: c -> a$ și \lstinline$pb :: c -> b$ astfel încât pentru orice alt obiect c' înzestrat cu două proiecții \lstinline$pa' :: c' -> a$ și \lstinline$pb' :: c' -> b$, există un unic morfism \lstinline$m :: c' -> c$ care factorizează aceste proiecții, adică:
\vspace{-1ex}
\onslide<2>
\begin{asciihs}
m . pa = pa'                                   m . pb = pb'

type Product a b = (a,b)
pa :: Product a b -> a
pa = fst
pb :: Product a b -> b
pb = snd

-- Factorizarea
factor  :: (c -> a) -> (c -> b) -> (c -> Product a b)
factor pa' pb' x = (pa' x, pb' x)

-- Avem
fst . factor pa' pb' = pa'        snd . factor pa' pb' = pb'
\end{asciihs}
\end{frame}


\begin{frame}[fragile]{Coproduse}
Coprodusul a două obiecte a și b este produsul lor în categoria duală, adică obiectul c înzestrat cu două injecții 
\lstinline$ia :: a -> ca$ și \lstinline$ib :: b -> c$ astfel încât pentru orice alt obiect c' înzestrat cu două injecții \lstinline$ia' :: a -> c'$ și \lstinline$ib' :: b -> c'$, există un unic morfism \lstinline$m :: c' -> c$ care factorizează aceste injecții, adică \lstinline$ia . m = ia'$ și \lstinline$ib . m = ib'$
\vspace{-1ex}
\onslide<2>
\begin{asciihs}
type Sum a b = Left a | Right b
ia :: a -> Sum a b
ia = Left
ib :: b -> Sum a b
ib = Right

-- Factorizarea
factor  :: (a -> c) -> (b -> c) -> (Sum a b -> c)
factor ia' ib' (Left x) = ia' x
factor ia' ib' (Right x) = ib' x
-- Avem
factor ia' ib' . ia = ia'        factor ia' ib' . ib = ib'
\end{asciihs}
\end{frame}

\end{document}



