\documentclass[addpoints,12pt,a4paper]{exam}
%,answers
\usepackage[none]{hyphenat} 
\usepackage{fullpage}
\usepackage{../tdefinition}
\pagestyle{empty}
%\header{Prenumele, numele și grupa: \enspace\hrulefill}{}{Subiect \thequestion}
\extrawidth{0.5in}
\newcommand{\bs}{\char`\\} \newcommand{\bt}{\char`\`}
\newcommand{\us}{\char`\_}

\begin{document}

\begin{center}
Programare declarativă---Examen scris \hfill  13 februarie 2016
\end{center}

\pointpoints{punct}{puncte}


\noindent\textbf{Atenție:} Tratați fiecare subiect pe câte o foaie separată!

\begin{questions}

\question[5] 
Recursivitate, descrieri de liste, map/filter/fold
\begin{parts}
\part[2] 
Să se scrie o funcție care pentru o listă de elemente oarecare și un element n inserează n înaintea fiecărui element al listei. Dacă lista e goală rezultatul va fi lista formată din elementul n.
\begin{asciihs}
  f :: a -> [a] -> [a] 
  f 'x' "ABCD" = "xAxBxCxD" 
  f 'a' "XY" = "aXaY" 
  f '-' "Hello" = "-H-e-l-l-o" 
  f '-' "" = "-" 
  f 0 [1,2,3,4,5] = [0,1,0,2,0,3,0,4,0,5] 
\end{asciihs}
Definiția poate folosi funcții elementare și recursie, dar nu descrieri de liste sau alte funcții din biblioteci.
\part[2]
Să se scrie o funcție care pentru o listă de numere întregi și un număr întreg strict pozitiv n determină cel mai mare element al listei cuprins între 0 și n. În cazul în care lista este goală, nu sunt elemente din intervalul [0,n] sau n  nu este strict pozitiv funcția va returna 0.
\begin{asciihs}
  g :: [Int] -> Int -> Int
  g [40,30,50,70,60] 100 = 70 
  g [-10,20,80,110] 100 = 80 
  g [-10,110] 100 = 0 
  g [] 100 = 0
  g [-10,20] -100 = 0 
\end{asciihs}
Definiția poate folosi funcții elementare, descrieri de liste și funcții din biblioteci (e.g., zip), dar nu recursie.

\part[1] Să se scrie o funcție, care primește ca parametru o listă de numere întregi și verifică dacă toate elementele din listă care se află în intervalul [-10,10] sunt divizibile cu 3.
\begin{asciihs}
  h :: [Int] -> Bool
  h[2, 5, 12, -2, 6, -9] = False
  h[6,-13, -9, 3, -3, 14] = True
  h[-14, 23] = True
\end{asciihs}
Definitia poate folosi una sau mai multe din funcțiile:
\begin{asciihs}
  map :: (a -> b) -> [a] -> [b]
  filter :: (a -> Bool) -> [a] -> [a]
  foldr :: (a -> b -> b) -> b -> [a] -> b
\end{asciihs}
Puteți folosi și funcții elementare, dar nu recursie, descrieri de liste sau alte funcții din biblioteci.
\end{parts}
\newpage
\noindent\textbf{Atenție:} Tratați fiecare subiect pe câte o foaie separată!
\question[2] 
Introducem un tip de date ce reprezinta o colectie de puncte (o tabela). 
\begin{asciihs}
type Point =(Int, Int)

data Points = Rectangle Point Point
              Union Points Points
              Difference Points	Points 
\end{asciihs}
			  
			  
Tabela incepe cu punctul (0,0) stanga jos. 
Constructorul Rectangle selecteaza toate punctele dintr-o forma rectangulara.
De pilda, Rectangle (0,0) (2,1) da colturile din stanga jos si dreapta sus ale unui dreptunghi si include punctele (0,0) ; (1,0) ;(2,0) ; (0,1) ; (1,1) ; (2,1)

Union combina doua colectii de puncte iar Difference contine acele puncte care sunt in prima colectie dar nu sunt in a doua. 			  

\begin{parts}
\part[1] Scrieti o functie \lstinline$perimeter$ care calculeaza perimetrul celui mai mic dreptunghi care cuprinde complet o colectie de puncte.

\begin{asciihs}
   perimeter :: Points -> Int 
\end{asciihs}

\part[1] Scrieti o functie distance care calculeaza distanta dintre doua colectii de puncte ca reprezentand distanta intre coltul dreapta-sus al dreptunghiului minimal care cuprinde prima colectie si coltul stanga-jos al dreptunghiului minimal care cuprinde cea de-a doua colectie.

\begin{asciihs}
  distance :: Points -> Points -> Int
\end{asciihs}

\end{parts}
\noindent\textbf{Atenție:} Tratați fiecare subiect pe câte o foaie separată!

\question[2]
Se dă următoarea definiție:

\begin{asciihs}
  instance Functor ((->) r) where
    fmap f g = (\x -> f (g x))

  instance Monad ((->) r) where
    return x = \_ -> x
    h >>= f = \w -> f (h w) w
\end{asciihs}

\textbf{Cerințe:}
\begin{parts}
\part[1] Să se verifice dacă fmap satisface axiomele de functor:
\begin{asciihs}
  fmap id  ==  id
  fmap (f . g)  ==  fmap f . fmap g
\end{asciihs}
\part[\half] Să se verifice dacă return și \lstinline$>>=$ satisfac axiomele monadei:
\begin{asciihs}
  return a >>= f == f a
  m >>= return == m
  (m >>= f) >>= g == m >>= (\x -> f x >>= g)
\end{asciihs}
\part[\half] Să se verifice dacă instanța de functor e compatibilă cu cea de monadă:
\begin{asciihs}
  fmap f ma  =  ma >>= return . f
\end{asciihs}
\noindent\textbf{Atenție:} Tratați fiecare subiect pe câte o foaie separată!
\end{parts}
\end{questions}
Se acordă un punct din oficiu.
\end{document}