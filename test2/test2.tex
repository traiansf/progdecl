\documentclass[addpoints,12pt,a4paper]{exam}
%,answers
\usepackage[none]{hyphenat} \usepackage{fullpage}
\usepackage{../tdefinition}
%\usepackage{fancyhdr}
%\pagestyle{fancy}
%\lhead{}
%}
%\rhead{Page \thepage}

\newcommand{\bs}{\char`\\} \newcommand{\bt}{\char`\`}
\newcommand{\us}{\char`\_}
\pagestyle{empty}

\begin{document}

\begin{center}

\end{center}

\pointpoints{punct}{puncte}

\begin{questions}

\question[3]
Definim un tip de date pentru a reprezenta o rețea de mulțimi de puncte în plan:
\begin{asciihs}
  type Punct = (Int,Int)
  data Multime = X 
               | Y 
               | DX Int Multime 
               | DY Int Multime 
               | U Multime Multime
\end{asciihs}
Planul este centrat în punctul $(0,0)$.  Prima coordonată a unui punct (coordonata $x$) reprezintă distanța pe orizontală de la origine iar a doua (coordonata $y$) reprezintă distanța pe verticală.  Prin convenție, coordonatele $x$ cresc spre dreapta, în timp ce coordonatele $y$ cresc în sus.  

Constructorul $X$ reprezintă mulțimea punctelor de pe axa $x$, adică punctele care au coordonata $y$ zero:
$$\cdots (-2,0) (-1,0) (0,0) (1,0) (2,0) \cdots$$

Constructorul $Y$ reprezintă mulțimea punctelor de pe axa $y$, adică punctele care au coordonata $x$ zero:
$$\cdots (0,-2) (0,-1) (0,0) (0,1) (0,2) \cdots$$

Constructorul \lstinline$DX dx p$, unde $dx$ este un întreg și $p$ e o mulțime de puncte, reprezintă punctele $p$ deplasate la dreapta cu $dx$. De exemplu, \lstinline$DX 2 Y$ are ca rezultat axa $y$ deplasată cu două unități la dreapta:
$$\cdots (2,-2) (2,-1) (2,0) (2,1) (2,2) \cdots$$

Observație 1:  \lstinline$DX dx X$ și $X$ denotă aceeași mulțime de puncte, deoarece prin deplasarea axei $x$ pe orizontală se obține tot axa $y$.

Observație 2: Un punct $(x,y)$ aparține lui \lstinline$DX dx p$ dacă și numai dacă punctul $(x-dx,y)$ aparține lui $p$.

Constructorul \lstinline$DY dy p$, unde $dy$ este un întreg și $p$ e o mulțime de puncte, reprezintă punctele $p$ deplasate în sus cu $dx$. De exemplu, \lstinline$DY 3 X$ are ca rezultat axa $X$ deplasată cu două unități în sus:
$$\cdots (-2,3) (-1,3) (0,3) (1,3) (2,3) \cdots$$

Observație 1:  \lstinline$DY dy Y$ și $Y$ denotă aceeași mulțime de puncte, deoarece prin deplasarea axei $y$ pe verticală se obține tot axa $y$.

Observație 2: Un punct $(x,y)$ aparține lui \lstinline$DY dy p$ dacă și numai dacă punctul $(x,y-dy)$ aparține lui $p$.

Constructorul \lstinline$U p q$, unde $p$ și $q$ sunt mulțimi de puncte, reprezintă reuniunea punctelor din $p$ și $q$.  De exemplu,
\lstinline$U (U X Y) (U (DY 3 X) (DX 2 Y))$
reprezintă mulțimea de puncte de forma

$\cdots (-2,0) (-1,0) (0,0) (1,0) (2,0) \cdots
\cdots (0,-2) (0,-1) (0,0) (0,1) (0,2) \cdots$

\hfill$\cdots (-2,3) (-1,3) (0,3) (1,3) (2,3) \cdots
\cdots (2,-2) (2,-1) (2,0) (2,1) (2,2) \cdots
$
\newpage
Cerințe:
\begin{parts}
\part[1\half] Scrieți o funcție \lstinline$apartine :: Punct -> Multime -> Bool$ care determină dacă un punct aparține unei mulțimi de puncte date.  De exemplu:
\begin{asciihs}
  apartine (3,0) X == True
  apartine (0,1) Y == True
  apartine (3,3) (DY 3 X) == True
  apartine (2,1) (DX 2 Y) == True
  apartine (3,0) (U X Y) == True
  apartine (0,1) (U X Y) == True
  apartine (3,3) (U (DY 3 X) (DX 2 Y)) == True
  apartine (2,1) (U (DY 3 X) (DX 2 Y)) == True
  apartine (3,0) (U (U X Y) (U (DX 2 Y) (DY 3 X))) == True
  apartine (0,1) (U (U X Y) (U (DX 2 Y) (DY 3 X))) == True
  apartine (3,3) (U (U X Y) (U (DX 2 Y) (DY 3 X))) == True
  apartine (2,1) (U (U X Y) (U (DX 2 Y) (DY 3 X))) == True
  apartine (1,1) X == False
  apartine (1,1) Y == False
  apartine (1,1) (DY 3 X) == False
  apartine (1,1) (DX 2 Y) == False
  apartine (1,1) (U X Y) == False
  apartine (1,1) (U X Y) == False
  apartine (1,1) (U (DY 3 X) (DX 2 Y)) == False
  apartine (1,1) (U (DY 3 X) (DX 2 Y)) == False
  apartine (1,1) (U (U X Y) (U (DX 2 Y) (DY 3 X))) == False
\end{asciihs}
\part[1\half] Scrieți o funcție \lstinline$nrAxe :: Multime -> Int$ care numără de câte ori apare X sau Y în descrierea unei mulțimi de puncte. Fiecare axă trebuie numărată câte o dată pentru fiecare apariție a sa. De exemplu:
\begin{asciihs}
  nrAxe X == 1
  nrAxe Y == 1
  nrAxe (U X Y) == 2
  nrAxe (U (DY 3 X) (DX 2 Y)) == 2
  nrAxe (U (U X Y) (U (DX 2 Y) (DY 3 X))) == 4
  nrAxe (U (U X Y) X) == 3 
\end{asciihs}
\end{parts}

\question[3] 
Se consideră următorul tip algebric de date:
\begin{asciihs}
  data Expr = Const Int
            | Neg Expr
            | Expr :+: Expr
            | Expr :*: Expr
            
  data Op = NEG | PLUS | TIMES

  data Atom = AConst Int | AOp Op

  type Polish = [Atom]
\end{asciihs}

\newpage \textbf{Cerințe:}
\begin{parts}
\part[1\half] 
Să se scrie o funcție \lstinline$fp :: Expr -> Polish$ care asociază unei expresii aritmetice date scrierea ei în forma poloneză: o listă de Atomi, obținută prin parcurgerea în preordine a arborelui asociat expresiei (operațiile precedând reprezentărilor operanzilor).
Exemple:
\begin{itemize}
\item forma poloneză a expresiei $5 * 3$ este $*\, 5\, 3$ \\
  \lstinline$fp (Const 5 :*: Const 3) = [AOp TIMES, AConst 5, AConst 3]$

\item forma poloneză a expresiei $−(7 * 3)$ este  $-\, *\, 7\, 3$ \\
  \lstinline$ fp (Neg (Const 7 :*: Const 3)) = [AOp Neg, AOp TIMES, AConst 7, AConst 3]$

\item forma poloneză a expresiei $(5 + −3) * 17$ este $*\,+\,5\,-\,3\, 17$\\
  \lstinline$fp ((Const 5 :+: Neg (Const 3)) :*: Const 17)$\\
  \lstinline$  = [AOp TIMES, AOp PLUS, AConst 5, AOp NEG, AConst 3, AConst 17]$

\item forma poloneză a expresiei $(15 + (7 * (2 + 1))) * 3$ este $*\, +\, 15\, *\, 7\, +\, 2\, 1\, 3$\\
  \lstinline$fp ((Const 15 :+: (Const 7 :*: (Const 2 :+: Const 1))) :*: Const 3)$\\
  \lstinline$  = [AOp TIMES, AOp PLUS, AConst 15, AOp TIMES, AConst 7,$\\ 
  \lstinline$     AOp PLUS, AConst 2, AConst 1, AConst 3]$

\end{itemize}
\part[1\half] Definiți o funcție \lstinline$rfp :: Polish -> Maybe Expr$ 
astfel încât \lstinline$rfp . fp = Just . id$
\end{parts}


\question[2] 
Introducem un tip de date ce reprezinta o colectie de puncte (o tabela). 
\begin{asciihs}
type Point =(Int, Int)

data Points = Rectangle Point Point
            | Union Points Points
            | Difference Points Points 
\end{asciihs}
			  
			  
Tabela incepe cu punctul (0,0) stanga jos. 
Constructorul Rectangle selecteaza toate punctele dintr-o forma rectangulara.
De pilda, Rectangle (0,0) (2,1) da colturile din stanga jos si dreapta sus ale unui dreptunghi si include punctele (0,0) ; (1,0) ;(2,0) ; (0,1) ; (1,1) ; (2,1)

Union combina doua colectii de puncte iar Difference contine acele puncte care sunt in prima colectie dar nu sunt in a doua. 			  

\begin{parts}
\part[1] Scrieti o functie \lstinline$perimeter$ care calculeaza perimetrul celui mai mic dreptunghi care cuprinde complet o colectie de puncte.

\begin{asciihs}
   perimeter :: Points -> Int 
\end{asciihs}

\part[1] Scrieti o functie distance care calculeaza distanta dintre doua colectii de puncte ca reprezentand distanta intre coltul dreapta-sus al dreptunghiului minimal care cuprinde prima colectie si coltul stanga-jos al dreptunghiului minimal care cuprinde cea de-a doua colectie.

\begin{asciihs}
  distance :: Points -> Points -> Int
\end{asciihs}

\end{parts}

\question[3] 
Introducem un tip de date ce reprezinta o expresie booleana formată din literali / variabile (Lit), negație (Not), conjuncție (And), disjuncție (Or) și implicație (:->:), in care conjuncțiile si disjuncțiile au un număr arbitrar de termeni. 
\begin{asciihs}
data Exp = Lit String
         | Not Exp
         | And [Exp]
         | Or [Exp]
         | Exp :->: Exp
  deriving (Show)
\end{asciihs}

Un atom este fie un literal (Lit), fie negația unui literal. O expresie este în formă normală disjunctivă dacă este o disjuncție de conjuncții de atomi.

Să se scrie o funcție care dată fiind o expresie are ca rezultat forma normală disjunctivă a acelei expresii.

\begin{asciihs}
   dnf :: Exp -> Exp
   dnf ((Lit "a" :->: Lit "b") :->: Lit "c")
     = Or [And [Lit "a",Not (Lit "b")],And [Lit "c"]]
   dnf (Lit "a" :->: (Lit "b" :->: Lit "c"))
     = Or [And [Not (Lit "b")],And [Lit "c"],And [Not (Lit "a")]]
\end{asciihs}

Indicație---puteți folosi următoarele identități:
\begin{itemize}
\item $a \rightarrow b \equiv \neg a \vee b$
\item $\neg \neg a = a$
\item $\neg\bigwedge (a_1,\ldots,a_n) = \bigvee(\neg a_1,\ldots,\neg a_n)$
\item $\neg\bigvee (a_1,\ldots,a_n) = \bigwedge(\neg a_1,\ldots,\neg a_n)$
\item $\bigwedge{\left(a_1,\ldots,a_i,\bigwedge (b_1,\ldots,b_m),a_{i+1},\ldots,a_n\right)}
= \bigwedge (a_1,\ldots,a_n,b_1,\ldots,b_m)$
\item $\bigvee{\left(a_1,\ldots,a_i,\bigvee (b_1,\ldots,b_m),a_{i+1},\ldots,a_n\right)}
= \bigvee (a_1,\ldots,a_n,b_1,\ldots,b_m)$
\item $\bigwedge{\left(a_1,\ldots,a_i,\bigvee (b_1,\ldots,b_m),a_{i+1},\ldots,a_n\right)}
= \bigvee{\left(\bigwedge(b_1,a_1,\ldots,a_n),\ldots,\bigwedge(b_m,\ldots,a_n)\right)}$
\end{itemize}

\end{questions}
\end{document}
