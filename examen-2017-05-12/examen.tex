\documentclass[addpoints,12pt,a4paper]{exam}
%,answers
\usepackage[none]{hyphenat} 
\usepackage{fullpage}
\usepackage{../tdefinition}
\pagestyle{empty}
%\header{Prenumele, numele și grupa: \enspace\hrulefill}{}{Subiect \thequestion}
\extrawidth{0.5in}
\newcommand{\bs}{\char`\\} \newcommand{\bt}{\char`\`}
\newcommand{\us}{\char`\_}

\begin{document}

\begin{center}
Programare declarativă---Examen scris \hfill  12 mai 2017
\end{center}

\pointpoints{punct}{puncte}

\begin{questions}

\question[3\half]{Elemente avansate de programare funcțională.}

Introducem un tip de date pentru a reprezenta arbori binari ai căror frunze sunt etichetate cu valori întregi, definiți astfel:
\begin{asciihs}
data Tree = Empty
   | Leaf Int
   | Node Tree Tree
\end{asciihs}
Exemple de arbori definiți cu reprezentarea de mai sus:

\begin{asciihs}
t1 = Empty
t2 = Node (Leaf 1)
          Empty
t3 = Node (Node (Node (Leaf 3)
                      Empty)
                (Leaf 1))
          (Node Empty
                (Node (Leaf 3)
                      (Leaf 5)))
t4 = Node (Node (Node Empty
                      Empty)
                (Leaf 1))
          (Node Empty
                (Node Empty
                      Empty))
\end{asciihs}
\begin{parts}
\part[1] Scrieți o funcție \lstinline$adancime :: Tree -> Int$ care calculează adâncimea maximă a unei frunze din arbore. De exemplu, pentru exemplele de mai sus:
\begin{asciihs}
adancime t1 = 0
adancime t2 = 2
adancime t3 = 4
adancime t4 = 3  -- deoarece la adancimea 4 nu sunt frunze
\end{asciihs}
Rezultatul trebuie să fie 0 pentru toți arborii care nu au frunze.

{\bf Indicație:} Arborii de forma \lstinline$Node t1 t2$ care nu conțin frunze trebuie tratați special.
\part[1\half] 
Scrieți o funcție \lstinline$adanci1 :: Tree -> [Int]$ care calculează lista valorilor asociate frunzelor de adâncime maximă, in ordinea în care acestea apar.
Folosiți funcția \lstinline$adancime$ pentru a calcula adancimea maximă, apoi calculați lista valorilor de la acea adâncime.

Pentru exemplele de mai sus:
\begin{asciihs}
adanci1 t1 = []
adanci1 t2 = [1]
adanci1 t3 = [3,3,5]
adanci1 t4 = [1]
\end{asciihs}

\part[1] Scrieți o funcție \lstinline$adanci2 :: Tree -> [Int]$  care calculează lista valorilor asociate frunzelor de adâncime maximă, in ordinea în care acestea apar, folosind o altă metodă: traversați arborele, aplicănd \lstinline$adanci2$ recursiv doar acelor subarbori care au frunze de adâncime maximă, folosind funcția \lstinline$adancime$ pentru ambii subarbori. Rezultatul ar trebui să fie același ca pentru \lstinline$adanci1$. Nu puteți folosi \lstinline$adanci1$ în rezolvare.
\end{parts}
\question[3\half]
{Elemente de bază de programare funcțională}
\begin{parts}
\part[1]
Scrieți o funcție \lstinline$p :: [String] -> Int$ care, dată fiind o listă de șiruri de caractere, însumează lungimile cuvintelor din listă care au fost abreviate. Pentru această problemă, o abreviere este orice șir de caractere care conține simbolul '.' (punct). De exemplu:
\begin{asciihs}
p ["Dr.","Who","crossed","the","ave."] == 7
p ["the","sgt.","opened","the","encl.","on","Fri.","pm"] == 13
p [] == 0
p ["John","Doe,","Ph.D"] == 4
p ["no","abbrevs","4U"] == 0
\end{asciihs}
Folosiți {\em funcții elementare}, {\em descrieri de liste} și {\em funcții din biblioteci}, dar nu {\em recursie}.

\part[1\half]
Scrieți o a doua funcție \lstinline$q :: [String] -> Int$ care se comportă la fel ca \lstinline$p$, folosind doar {\em funcții elementare}, {\em recursie} și {\em funcții din biblioteci}, dar nu  {\em descrieri de liste}.

\part[1]
Scrieți o a treia funcție \lstinline$r :: [String] -> Int$ care se comportă la fel ca \lstinline$p$, de data aceasta folosind următoarele funcții de ordin înalt din biblioteca standard:
\begin{asciihs}
map :: (a -> b) -> [a] -> [b]
filter :: (a -> Bool) -> [a] -> [a]
foldr :: (a -> b -> b) -> b -> [a] -> b
\end{asciihs}
Nu puteți folosi {\em recursie} și {\em descrieri de liste}.
\end{parts}

\question[2]{Elemente specifice limbajului Haskell.}

Se dă următoarea secvență de declarații în Haskell:

\begin{asciihs}
type Id = String
data AExp = Lit Int | Var Id | AExp :+: AExp | AExp :<=: AExp
data Stmt = Decl Id | If AExp Stmt Stmt | Id := AExp | Stmt :; Stmt | Skip
type Mem = [(Id, Int)]
eval :: Stmt -> Mem
\end{asciihs}

Să se completeze definiția funcției \lstinline$eval$ pentru a putea calcula starea memoriei (Mem) în urma execuției instrucțiunii date ca argument.  Observatie: Puteți defini oricâte funcții ajutătoare.

\lstinline$Decl$ inițializează o nouă varibilă în memorie (dacă nu exista deja), cu valoarea 0.
\lstinline$If$ are aceeași semantică ca în C (dacă valoarea expresiei e diferită de 0 se execută prima instrucțiune, dacă nu cea de-a doua). Operatorul \lstinline$:;$ este operatorul de compunere secvențială a instrucțiunilor; semantica lui este: se execută prima instrucțiune mai întâi, apoi a doua instrucțiune se execută în starea de după execuția primei instrucțiuni. \lstinline$Skip$ este instrucțiunea vidă. 

Punctaj: 2 puncte din care 0,5 rezolvarea, + 1 rezolvarea folosind monade + 0,5 rezolvarea folosind monada predefinită cea mai potrivită.
\end{questions}

Se acordă un punct din oficiu.
\end{document}
\question[24] 
Recursivitate, descrieri de liste, map/filter/fold
\begin{parts}
\part[12]
Scrieți o funcție \lstinline$f :: String -> String$ care repetă fiecare caracter succesiv din șirul de intrare odată în plus față de caracterul precedent, începând cu o primă apariție a primului caracter. De exemplu:
\begin{asciihs}
f "abcde"  == "abbcccddddeeeee"
f "ZYw"    == "ZYYwww"
f ""       == ""
f "Inf1FP" == "Innfff1111FFFFFPPPPPP"
\end{asciihs}
Folosiți {\em funcții elementare}, {\em descrieri de liste} și {\em funcții din biblioteci}, dar nu {\em recursie}.
\part[12]
Scrieți o a doua funcție \lstinline$g :: String -> String$ care se comportă la fel ca \lstinline$f$, folosind doar {\em funcții elementare}, {\em recursie} și {\em funcții din biblioteci}, dar nu  {\em descrieri de liste}.
\end{parts}

