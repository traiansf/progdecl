\documentclass[xcolor=pdftex,romanian,colorlinks]{beamer}

\usepackage[export]{adjustbox}
\usepackage{../tslides}
\usepackage[all]{xy}
\usepackage{pgfplots}
\usepackage{flowchart}
\usetikzlibrary{arrows,positioning,calc}
\lstset{language=Haskell}
\lstset{escapeinside={(*@}{@*)}}

\AtBeginSection[]{
  \begin{frame}
  \vfill
  \centering
  \begin{beamercolorbox}[sep=8pt,center,shadow=true,rounded=true]{title}
    \usebeamerfont{title}\insertsectionhead\par%
  \end{beamercolorbox}
  \vfill
  \end{frame}
}


\title[PD---Monade]{Programare declarativă\thanks{bazat pe cursul \emph{Informatics 1: Functional Programming} de la \emph{University of Edinburgh}}}

\subtitle{Monade Standard}

\begin{document}
\begin{frame}
  \titlepage
\end{frame}

\section{Evaluare cu efecte laterale}

\subsection{Sintaxă abstractă}

\begin{frame}[fragile]{Lambda calcul cu întregi}
{Sintaxă}
\lstinputlisting[firstline=8,lastline=14]{var0.hs}
\end{frame}


\begin{frame}[fragile]
{Valori și medii de evaluare}
\lstinputlisting[firstline=37,lastline=46]{var0.hs}
\end{frame}

\begin{frame}[fragile]
{Evaluare}{Variabile și valori}
\lstinputlisting[linerange={48-52,61-65}]{var0.hs}
\end{frame}

\begin{frame}[fragile]
{Evaluare}{Adunare}
\lstinputlisting[linerange={53-56,66-69}]{var0.hs}
\end{frame}

\begin{frame}[fragile]
{Evaluare}{Aplicarea funcțiilor}
\lstinputlisting[linerange={57-60,70-73}]{var0.hs}
\end{frame}

\begin{frame}[fragile]{Testarea interpretorului}
\lstinputlisting[linerange={75-76}]{var0.hs}

unde

\lstinputlisting[linerange={5-5}]{var0.hs}

este o funcție definită special pentru fiecare tip de efecte laterale dorit.
\end{frame}

\begin{frame}[fragile]{Interpretare în monada Identitate}
\lstinputlisting[linerange={3-6}]{var0.hs}

Unde monada \lstinline$Identity$ capturează transformarea identitate:

\begin{asciihs}
newtype Identity a = Identity { runIdentity :: a }
instance Monad Identity where
    return   = Identity
    m >>= k  = k (runIdentity m)
\end{asciihs}
\structure{Observație:} Obținem interpretorul standard discutat în cursurile trecute
\end{frame}

\begin{frame}[fragile]{Interpretare în monada Optiune}
Putem renunța la valoarea \lstinline$Wrong$ folosind monada \lstinline$Maybe$
\lstinputlisting[linerange={2-6}]{var1.hs}

Putem acum înlocui rezultatele \lstinline$Wrong$ cu \lstinline$Nothing$
\lstinputlisting[linerange={60-62,65-66,69-70}]{var1.hs}
\end{frame}

\begin{frame}[fragile]{Interpretare în monada Either String}
Putem nuanța erorile folosind monada \lstinline$Either String$
\lstinputlisting[linerange={2-3,5-6}]{var2.hs}

Putem acum înlocui rezultatele \lstinline$Wrong$ cu valori \lstinline$Left$
\lstinputlisting[linerange={60-62,65-67,70-71}]{var2.hs}
\end{frame}

\begin{frame}[fragile]{Monada Stare}{Control.Monad.State}
\begin{asciihs}
newtype State s a = State { runState :: s -> (a, s) }

instance Monad (State s) where
  return x = State (\ s -> (x,s))
  m >>= k  = State (\ s -> 
    let (x, s') = runState m s
     in runState (k x) s'

get :: State s s                 -- produce starea curenta
get = State (\ s -> (s,s))     

put :: s -> State s ()           -- schimba starea curenta
put s = State (\ _ -> ((), s))

modify :: (s -> s) -> State s () -- modfica starea
modify f = State (\ s -> ((), f s))
\end{asciihs}
\end{frame}

\begin{frame}[fragile]{Interpretare în monada Stare}{Adăugarea unui contor de intstrucțiuni}
\begin{asciihs}
data Term = ... | Count
\end{asciihs}

\lstinputlisting[linerange={3-4,12-13}]{var3.hs}

\vspace{-1ex}\lstinputlisting[linerange={78-78,82-82}]{var3.hs}
unde
\lstinputlisting[linerange={5-6}]{var3.hs}

Iar evaluarea lui \lstinline$Count$ se face astfel:
\lstinputlisting[linerange={68-70}]{var3.hs}
\end{frame}

\begin{frame}[fragile]{Monoizi}{Data.Monoid.Monoid}
\begin{asciihs}
class Monoid m where
    mempty  :: m
    mappend :: m -> m -> m
\end{asciihs}

\begin{block}{Monoidul listelor}
\begin{asciihs}
instance Monoid [a] where
    mempty  = []
    mappend = (++)
\end{asciihs}
\end{block}
\begin{block}{Monoidul conjunctiv\hfill Data.Monoid.All}
\vspace{-2ex}
\begin{asciihs}
newtype All = All { getAll :: Bool }

instance Monoid All where
        mempty = All True
        All x `mappend` All y = All (x && y)
\end{asciihs}
\end{block}
\end{frame}

\begin{frame}[fragile]{Monada Writer}{Control.Monad.Writer}
Este folosită pentru a acumula (logging) informatie produsă în timpul execuției
\begin{asciihs}
newtype Writer w a = Writer { runWriter :: (a, w) }

instance Monoid w => Monad (Writer w) where
  return x = Writer (x, mempty)
  ma >>= k = Writer $
    let (x, w)   = runWriter ma
        (x', w') = runWriter (k x)
     in (x', w `mappend` w')

tell :: w -> Writer w ()
tell w = Writer ((), w)
\end{asciihs}
\end{frame}

\begin{frame}[fragile]{Interpretare în monada Writer}{Adăugarea unei instrucțiuni de afișare}
\begin{asciihs}
data Term = ... | Out Term
\end{asciihs}

\lstinputlisting[linerange={3-7}]{var4.hs}

\lstinputlisting[linerange={62-65}]{var4.hs}

\begin{itemize}
\item \lstinline$Out t$ se evaluează la valoarea lui \lstinline$t$ 
\item cu efectul lateral de a adăuga valoarea la șirul de ieșire.
\end{itemize}
\end{frame}

\begin{frame}[fragile]{Interpretare în monada listelor}{Adăugarea unei instrucțiuni nedeterministe}
\begin{asciihs}
data Term = ... | Amb Term Term | Fail
\end{asciihs}

\lstinputlisting[linerange={2-5}]{var5.hs}

\lstinputlisting[linerange={61-62}]{var5.hs}

\begin{asciihs}
Main> test (App (Lam "x" (Var "x" :+: Var "x")) (Amb (Con 1) (Con 2)))
"[2,4]"
\end{asciihs}
\end{frame}

\begin{frame}[fragile]{Monada Reader}{Control.Monad.Reader}
Face accesibilă o memorie imutabilă (environment)
\begin{asciihs}
newtype Reader r a = Reader { runReader :: r -> a }

instance Monad (Reader r) where
  return x = Reader (\ _ -> x)
  ma >>= k = Reader $ \ r ->
    let x = runReader ma r
     in runReader (k x) r

-- obtine memoria
ask :: Reader r r
ask = Reader (\r -> r)

-- modifica local memoria
local :: (r -> r) -> Reader r a -> Reader r a
local f ma = Reader $ \r -> runReader ma (f r)
\end{asciihs}
\end{frame}  

\begin{frame}[fragile]{Interpretare în monada Reader}{Eliminarea argumentului \lstinline$Environment$ -- expresii de bază și lookup}
\lstinputlisting[linerange={3-6}]{var6.hs}

\lstinputlisting[linerange={48-50,63-69}]{var6.hs}
\end{frame}

\begin{frame}[fragile]{Interpretare în monada Reader}{Eliminarea argumentului \lstinline$Environment$ -- operatori binari și funcții}
\lstinputlisting[linerange={55-62}]{var6.hs}

\lstinputlisting[linerange={51-54}]{var6.hs}
\end{frame}

\end{document}



