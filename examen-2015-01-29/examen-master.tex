\documentclass[addpoints,12pt,a4paper,answers]{exam}
%
\usepackage[none]{hyphenat} \usepackage{fullpage}
\usepackage{../tdefinition}
%\usepackage{fancyhdr}
%\pagestyle{fancy}
%\lhead{}
%}
%\rhead{Page \thepage}

\newcommand{\bs}{\char`\\} \newcommand{\bt}{\char`\`}
\newcommand{\us}{\char`\_}
\pagestyle{empty}


\newcommand{\getHeight}[1]{%
%\newdimen\height
%\setbox0=\vbox{#1}
%\height=\ht0 \advance\height by \dp0
%\vspace*{1.1\height}
%#1%
}

\newenvironment{solutie}{\par\hspace*{-9em}\begin{minipage}{.98\paperwidth}
\hrulefill {\bf Rezolvare} \hrulefill}{\hrulefill\end{minipage}}

\begin{document}

\begin{center}

%\makebox[\textwidth]{Prenumele, numele și grupa: \enspace\hrulefill}
%\vspace{0.1in}

Programare declarativă---Examen \hfill  29 ianuarie 2016 \\ \ \\

\end{center}

\pointpoints{punct}{puncte}

\begin{questions}
\question[2] {\bf  IMP+REPEAT. }
Adăugați la limbajul IMP instrucțiunea $\terminal{repeat} e \terminal{until} c$ care ia ca argument o instrucțiune $e$ și o expresie booleană $c$, cu semantica uzuală: se execută $e$, apoi se evaluează expresia $c$ la o valoare $b$; dacă  $b$ e $\Strue$, atunci rezultatul e instrucțiunea vidă; dacă $b$ e $\Sfalse$, atunci se revine la începutul instrucțiunii repetitive.
\begin{parts}
\part[\half] {\bf Semantică. } Scrieți regulile de deducție pentru execuția programelor folosind $\terminal{repeat}\_\terminal{until}\_$ 
\getHeight{
\begin{solutie}
(0,25p/regulă)

\[
\reg[SwitchS]{
\Ss
{\c{\terminal{switch}(e)\{\terminal{case} v_1: e_1; \terminal{case} v_2: e_2; \terminal{default:} d\},s}}
{\c{\terminal{switch}(e')\{\terminal{case} v_1: e_1; \terminal{case} v_2: e_2; \terminal{default:} d\},s'}}
}{
\Ss
{\c{e,s}}
{\c{e',s'}}
}{}
\]
\[
\reg[Switch1]{
\Ss
{\c{\terminal{switch}(v)\{\terminal{case} v_1: e_1; \terminal{case} v_2: e_2; \terminal{default:} d\},s}}
{\c{e_1,s}}}{}{v=v_1}
\]
\[
\reg[Switch2]{
\Ss
{\c{\terminal{switch}(v)\{\terminal{case} v_1: e_1; \terminal{case} v_2: e_2; \terminal{default:} d\},s}}
{\c{e_2,s}}}{}{v\neq v_1, v = v_2}
\]
\[
\reg[Switch3]{
\Ss
{\c{\terminal{switch}(v)\{\terminal{case} v_1: e_1; \terminal{case} v_2: e_2; \terminal{default:} d\},s}}
{\c{d,s}}}{}{v\neq v_1, v \neq v_2}
\]
\end{solutie}
}
\part[\half] {\bf Tipuri. } Scrieți regula de tipuri pentru $\terminal{repeat}\_\terminal{until}\_$
\getHeight{
\begin{solutie}
(0,2p/câte au tipul la fel ca e sau câte au tipul la fel ca rezultatul)

\[\reg[tSwitch]{\tjud{\terminal{switch}(e)\{\terminal{case} v_1: e_1; \terminal{case} v_2: e_2; \terminal{default:} d\}}{T'}}{\tjud{e}{T}\si\tjud{v_1}{T}\si\tjud{e_1}{T'}\si\tjud{v_2}{T}\si\tjud{e_2}{T'}\si\tjud{d}{T'}}{}\]
\end{solutie}
}
\part[1] {\bf Siguranță. } Arătați că sistemul de tranziție IMP+REPEAT are proprietatea de conservare a tipului.
\getHeight{
\begin{solutie}

Presupun proprietatea e adevărată pentru toate premizele regulii {\sc tSwitch} și o demonstrez pentru concluzia ei:

(0,2p) Mai întâi, $\terminal{switch}(e)\{\terminal{case} v_1: e_1; \terminal{case} v_2: e_2; \terminal{default:} d\}$ nu este o valoare.  Fie $s$  astfel încât $\Dom(\Gamma)\subseteq \Dom s$.  

(0,6)Dacă $e$ este valoare: dacă $e=v_1$ atunci aplicăm regula {\sc Switch1}; altfel, dacă $e=v_2$ aplicăm regula {\sc Switch2}; alftel aplicăm regula {\sc Switch3}.

(0,2)Dacă $e$ nu este valoare, din ipoteza de inducție, există $e'$,$s'$ astfel încât
$\Ss{\c{e,s}}{\c{e',s'}}$.  Putem deci aplica regula {\sc SwitchS}.

\end{solutie}
}
\end{parts}

\question[3\half] {\bf Funcții}
Se dă expresia $e$
\begin{asciiml}
let x = let x = (fun (x:bool) -> y x) x
        in (fun (y:int) -> y <= x) (let rec y : int = x + y in x)
in y x
\end{asciiml}
\begin{parts}
\part[1\half] {\bf Variabile. } Subliniați aparițiile libere și indicați legătura fiecărei apariții legate printr-o săgeată către parametrul care o leagă

\part[2] \label{tipFn}{\bf Tipuri. } Demonstrați că 
\[x:{\Sbool},y:{\Sbool \rightarrow \Sint} \vdash e : {\Sint}\]
\getHeight{
\begin{solutie}
(0,15p/ prima aplicare a unei reguli;  pentru fiecare regula din primele 4 nivele, afara de al doilea Fun -> 0,1p)
\[
\reg[tLet]{
\tjud[]{\Slet x = {\Sfun (x:{\Sint}) \rightarrow x \terminal{<=} 1} \Sin {\Sif (x\; 3) \Sthen (\Sfun (x:{\Sint}) \rightarrow x) \Selse x}}{?}
}{
\tjud[]{\Sfun (x:{\Sint}) \rightarrow x \terminal{<=} 1}{{\Sint}\rightarrow{\Sbool}}
\si 
\tjud[x:{\Sint}\rightarrow{\Sbool}]{\Sif (x\; 3) \Sthen (\Sfun (x:{\Sint}) \rightarrow x) \Selse x}{?}
}{}
\]

\[
\reg[tFun]{
\tjud[]{\Sfun (x:{\Sint}) \rightarrow x \terminal{<=} 1}{{\Sint}\rightarrow{\Sbool}}
}{
\reg[t$\leq$]{
\tjud[x:{\Sint}]{x \terminal{<=} 1}{\Sbool}
}{
\reg[tVar]{
\tjud[x:{\Sint}]{x}{\Sint}
}{\done}{}
\si
\reg[tInt]{
\tjud[x:{\Sint}]{1}{\Sint}
}{\done}{}
}{}
}{}
\]

Fie $\Gamma=x:{\Sint}\rightarrow{\Sbool}$
\[
\reg[tIf]{
\tjud{\Sif (x\; 3) \Sthen (\Sfun (x:{\Sint}) \rightarrow x) \Selse x}{?}
}{
\tjud{x\; 3}{\Sbool}
\si 
\reg[tFun]{
\tjud{\Sfun (x:{\Sint}) \rightarrow x}{{\Sint}\rightarrow{\Sint}}
}{
\reg[tVar]{
\tjud[x:{\Sint}]{x}{\Sint}
}{\done}{}
}{}
\si
\reg[tVar]{ 
\tjud{x}{{\Sint}\rightarrow{\Sbool}}
}{\done}{}
}{}
\]

\[\reg[tApp]{
\tjud{x\; 3}{\Sbool}
}{
\reg[tVar]{
\tjud{x}{{\Sint}\rightarrow{\Sbool}}
}{\done}{}
\si 
\reg[tInt]{
\tjud{3}{\Sint}
}{\done}{}
}{}
\]

\end{solutie}
}
%\part[1] {\bf Evaluare. } Efectuați primii 5 pași din evaluarea non-strictă a expresiei $e$. Pentru fiecare pas indicați pe săgeată axioma folosită. Pentru evaluarea non-strictă a expresiilor de tip $\Slet$ folosiți axioma:
%\[\reg[NSLet]{\Ss{\c{\Slet x = e_1 \Sin e_2,s}}{\c{e',s}}}{}{e'=e_2[e_1/x]}\]
\getHeight{
\begin{solutie}
0,25p pe tranziție
\[\begin{array}{cl}
&
\c{\Slet x = {\Sfun (x:{\Sint}) \rightarrow x \terminal{<=} 1} \Sin {\Sif (x\; 3) \Sthen (\Sfun (x:{\Sint}) \rightarrow x) \Selse x},\emptyset}
\\
\xrightarrow{\textsc{Let}} &
\c{\Sif (({\Sfun (x:{\Sint}) \rightarrow x \terminal{<=} 1})\; 3) \Sthen (\Sfun (x:{\Sint}) \rightarrow x) \Selse {\Sfun (x:{\Sint}) \rightarrow x \terminal{<=} 1},\emptyset}
\\
\xrightarrow[\textsc{IfS}]{\textsc{App}} &
\c{\Sif 3 \terminal{<=} 1 \Sthen (\Sfun (x:{\Sint}) \rightarrow x) \Selse {\Sfun (x:{\Sint}) \rightarrow x \terminal{<=} 1},\emptyset}
\\
\xrightarrow[\textsc{IfS}]{\textsc{Op}\leq} &
\c{\Sif {\Sfalse} \Sthen (\Sfun (x:{\Sint}) \rightarrow x) \Selse {\Sfun (x:{\Sint}) \rightarrow x \terminal{<=} 1},\emptyset}
\\
\xrightarrow{\textsc{If}} &
\c{\Sfun (x:{\Sint}) \rightarrow x \terminal{<=} 1,\emptyset}
\end{array}\]
\end{solutie}
}
\end{parts}
%\newpage


\question[3\half] {{\bf Referințe și date structurate} (DATA-IMP).}
Fie $e$ expresia:
\begin{asciiml}
(fun (a : {r:{e:int}}) -> a.r.e) {t = 0; r = {e = 0; i = ref false}}
\end{asciiml}

\begin{parts}

%\part[1] {\bf SubTipuri. } Probați (demonstrați) dacă:
%\begin{asciiml}
%{r:{}} ->  {}   <:  {c:bool; r:{d:int}} -> {}
%\end{asciiml}

%\getHeight{
%\begin{solutie}
%(0,15p/prima folosire a unei reguli; 0,05p pentru următoarele)
%
%\(\displaystyle
%\reg[sFun]{
%\{r:\{\}\}\rightarrow \{\} \stip \{c:{\Sbool}; r:\{d: {\Sint}\}\} \rightarrow \{\}
%}{\displaystyle
%\{c:{\Sbool}; r:\{d: {\Sint}\}\} \stip \{r:\{\}\}
%\si
%\reg[sRefl]{
%\{\} \stip \{\}
%}{\done}{}
%}{}
%\)\\\\
%\(\displaystyle
%\reg[sTran]{
%\{c:{\Sbool}; r:\{d:{\Sint}\} \stip \{r:\{\}\}
%}{
%\reg[sRecordOrder]{\displaystyle
%\{c:{\Sbool}; r:\{d:{\Sint}\} \stip \{r:\{d:{\Sint}; c:{\Sbool}\}
%}{\done}{}
%\si
%\{r:\{d:{\Sint}; c:{\Sbool}\} \stip \{r:\{\}\}
%}{}
%\)\\\\
%\(\displaystyle
%\reg[sTran]{
%\{r:\{d:{\Sint}; c:{\Sbool}\} \stip \{r:\{\}\}
%}{
%\reg[sRecordWidth]{\displaystyle
%\{r:\{d:{\Sint}; c:{\Sbool}\} \stip  \{r:\{d:{\Sint}\} 
%}{\done}{}
%\si  
%\reg[sRecordDepth]{\displaystyle
%\{r:\{d:{\Sint}\} \stip \{r:\{\}\}
%}{
%\reg[sRecordWidth]{\displaystyle
%\{d:{\Sint} \stip \{\}
%}{\done}{}
%}{}
%}{}
%\)
%
%\end{solutie}
%}

\part[2\half] Demonstrați că există un tip $T$ astfel încât $\tjud[]{e}{T}$.

\getHeight{
\begin{solutie}(0,15p/prima folosire a unei reguli; 0,05p pentru următoarele)

Fie $t=\{c:{\Sbool}; r:\{\}\} \rightarrow \{\}$

\(\displaystyle
\reg[tApp]{
\tjud[]{
(\Sfun (f : t) \rightarrow  f\; \{c={\Strue}; r=\{d=7 + 3\}\})\; {\Sfun (x:\{r:\{\}\}) \rightarrow  \{r=x.r\}}
}{\{\}}
}{
\tjud[]{\Sfun (f : t) \rightarrow  f\; \{c={\Strue}; r=\{d=7+3\}\}}{t \rightarrow \{\}}
\si
\tjud[]{\Sfun (x:\{r:\{\}\}) \rightarrow  \{r=x.r\}}{t}
}{}
\)\\\\
\(\displaystyle
\reg[tFun]{
\tjud[]{\Sfun (f : t) \rightarrow  f\; \{c={\Strue}; r=\{d=7+3\}\}}{t \rightarrow \{\}}
}{
\tjud[f:t]{f\; \{c={\Strue}; r=\{d=7+3\}\}}{\{\}}
}{}
\)\\\\
\(\displaystyle
\reg[tSubSUM]{
\tjud[]{\Sfun (x:\{r:\{\}\}) \rightarrow  \{r=x.r\}}{t}
}{
\reg[tFun]{
\tjud[]{\Sfun (x:\{r:\{\}\}) \rightarrow  \{r=x.r\}}{\{r:\{\}\}\rightarrow \{r:\{\}\}}
}{
\reg[tRecord]{
\tjud[x:\{r:\{\}\}]{\{r=x.r\}}{\{r:\{\}\}}
}{
\reg[tField]{\tjud[x:\{r:\{\}\}]{x.r}{\{\}}
}{\reg[tVar]{
\tjud[x:\{r:\{\}\}]{x}{\{r:\{\}\}}
}{\done}{}
}{}
}{}
}{}
}{}
\)\\\\
\(
\reg[tApp]{
\tjud[f:t]{f\; \{c={\Strue}; r=\{d=7\}\}}{\{\}}
}{
\reg[tVar]{
\tjud[f:t]{f}{t}
}{\done}{}
\si
\reg[tRecord]{
\tjud[f:t]{\{c={\Strue}; r=\{d=7\}\}}{\{c:{\Sbool}; r:\{d: {\Sint}\}\}}
}{
\reg[tBool]{
\tjud[f:t]{\Strue}{\Sbool}
}{\done}{}
\si
\reg[tRecord]{
\tjud[f:t]{\{d=7\}}{d:{\Sint}}
}{
\reg[tInt]{
\tjud[f:t]{7}{\Sint}
}{\done}{}
}{}
}{}
}{}
\)

\end{solutie}
}

\part[1] Scrieți arborii de demonstrație pentru fiecare pas din execuția {\bf non-strictă} a expresiei $e$ din configurația inițială $\c{e,\emptyset}$ până la o configurație finală.

\getHeight{
\begin{solutie}
(0,15p/primii doi pași; 0,2p al treilea)

\(\displaystyle
\reg[App]{
\Ss{\c{e,\emptyset}}
   {\c{(\Sfun (x:\{r:\{\}\}) \rightarrow  x.r) \; \{c={\Strue}; r=\{d=7\}\},\emptyset}}
   }{\done}{}
\)\\\\
\(\displaystyle
\reg[App]{
\Ss{{\c{(\Sfun (x:\{r:\{\}\}) \rightarrow  x.r) \; \{c={\Strue}; r=\{d=7\}\},\emptyset}}}
   {{\c{\{c={\Strue}; r=\{d=7\}\}.r,\emptyset}}}
}{\done}{}
\)\\\\
\(\displaystyle
\reg[Field]{
\Ss{{\c{\{c={\Strue}; r=\{d=7\}\}.r,\emptyset}}}
   {{\c{\{d=7\},\emptyset}}}
}{\done}{}
\)

\end{solutie}
}

\end{parts}
%\newpage

\end{questions}

%Ciornă
\end{document}
