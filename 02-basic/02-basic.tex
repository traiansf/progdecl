\documentclass[handout,xcolor=pdftex,romanian,colorlinks]{beamer}

\usepackage[export]{adjustbox}
\usepackage{../tslides}
\usepackage[all]{xy}
\usepackage{pgfplots}
\usepackage{flowchart}
\usetikzlibrary{arrows,positioning,calc}
\lstset{language=Haskell}


\AtBeginSection[]{
  \begin{frame}
  \vfill
  \centering
  \begin{beamercolorbox}[sep=8pt,center,shadow=true,rounded=true]{title}
    \usebeamerfont{title}\insertsectionhead\par%
  \end{beamercolorbox}
  \vfill
  \end{frame}
}


\title[PD---Baze]{Programare declarativă}
\subtitle{Funcții, liste, recursie}
\begin{document}
\begin{frame}
  \titlepage
\end{frame}

\begin{section}{Funcții și recursie}
\begin{frame}{Ce e o funcție?}
\onslide<2>
\begin{itemize}
\item DEX(online): Mărime variabilă care depinde de una sau de mai multe mărimi variabile independente
\vitem O rețetă pentru a obține ieșiri din intrări: 
\hfill \structure{„Ridică un număr la pătrat”}

\vitem O relație între intrări și ieșiri
      \hfill \structure{$\{(1,1), (2,4), (3,9), (4,16), \ldots\}$} 
      
\vitem O ecuație algebrică \hfill \structure{$ f(x) = x^2 $}
\vitem Un grafic reprezentând ieșirile pentru intrările posibile
 \pgfplotsset{height=5cm}
\begin{tikzpicture}
  \begin{axis}[ 
    xlabel=$x$,
    ylabel={$f(x) = x^2$}
  ] 
    \addplot[blue] {x^2}; 
  \end{axis}
\end{tikzpicture}      
\end{itemize}
\end{frame}

\begin{frame}{Tipuri de date}{pentru intrări/ieșiri ale funcțiilor}
\begin{itemize}
\item \structure{Integer}:  4, 0, -5
\vitem \structure{Float}: 3.14
\vitem \structure{Char}: 'a'
\vitem \structure{String}: "abc"
\vitem \structure{Bool}: True, False
\vitem<2> \structure{Picture}: \includegraphics[scale=.05, valign=t]{knight.png}, \includegraphics[scale=.05, valign=t]{bishop.png}
\end{itemize}
\end{frame}
\tikzset{>=latex} 
\begin{frame}[fragile]{Tipuri de funcții și aplicarea lor}
\begin{asciihs}
invert :: Picture -> Picture

knight :: Picture

invert knight
\end{asciihs}
\vfill
\begin{tikzpicture}
\node[inner sep=0pt] (wknight) at (0,0)
    {\includegraphics[width=.1\textwidth]{knight.png}};
\node (process1) at (3,0) [draw, process,
minimum width=1cm,
minimum height=1cm] {invert};
\node[inner sep=0pt] (bknight) at (6,0)
    {\includegraphics[width=.1\textwidth]{black-knight.png}};
\draw[->] (wknight) -> (process1);
\draw[->] (process1) -> (bknight);
\end{tikzpicture}
\end{frame}

\begin{frame}[fragile]{Compunerea funcțiilor}
\begin{asciihs}
beside :: Picture -> Picture -> Picture
flipV :: Picture -> Picture
invert :: Picture -> Picture

knight :: Picture

beside (flipV knight) (invert knight)
\end{asciihs}

\begin{tikzpicture}
\node[inner sep=0pt] (wknight) at (0,0)
    {\includegraphics[width=.1\textwidth]{knight.png}};
\node (process1) at (2,-1) [draw, process,
minimum width=1cm,
minimum height=1cm] {invert};
\node (process2) at (2,1) [draw, process,
minimum width=1cm,
minimum height=1cm] {flipV};
\node[inner sep=0pt] (bknight) at (4.5,-1)
    {\includegraphics[width=.1\textwidth]{black-knight.png}};
\node[inner sep=0pt] (fknight) at (4.5,1)
    {\includegraphics[width=.1\textwidth]{flop-knight.png}};
\node (process3) at (7,0) [draw, process,
minimum width=1cm,
minimum height=1cm] {beside};
\node[inner sep=0pt] (result) at (10,0)
    {
\includegraphics[width=.1\textwidth]{flop-knight.png}
    \includegraphics[width=.1\textwidth]{black-knight.png}    };
\draw[->] (wknight.350) -| (process1);
\draw[->] (process1) -> (bknight);
\draw[->] (wknight.10) -| (process2);
\draw[->] (process2) -> (fknight);
\draw[->] (process3) -> (result);
\draw[->] (bknight) |- (process3.190);
\draw[->] (fknight) |- (process3.170);
\end{tikzpicture}
\end{frame}

\begin{frame}[fragile]{Definirea unei funcții noi}
\begin{asciihs}
double :: Picture -> Picture
double p = beside (flipV p) (invert p)

double knight
\end{asciihs}

%\begin{tikzpicture}[remember picture]
%\node[inner sep=0pt] (wknight)
%    {\includegraphics[width=.1\textwidth]{knight.png}};
%\node[right=of wknight] (process4) {
%\begin{tikzpicture}
%\node (process1) [draw, process,
%minimum width=1cm,
%minimum height=1cm] {invert};
%\node[below=of process1)  (process2) [draw, process,
%minimum width=1cm,
%minimum height=1cm] {flipV};
%\node[inner sep=0pt, right=of process1] (bknight) at (4.5,-1)
%    {\includegraphics[width=.1\textwidth]{black-knight.png}};
%\node[inner sep=0pt, right=of process2] (fknight) at (4.5,1)
%    {\includegraphics[width=.1\textwidth]{flop-knight.png}};
%\node[right=of bknight] (process3) at (7,0) [draw, process,
%minimum width=1cm,
%minimum height=1cm] {beside};
%\end{tikzpicture}
%};
%\node[inner sep=0pt] (result) at (10,0)
%    {
%\includegraphics[width=.1\textwidth]{flop-knight.png}
%    \includegraphics[width=.1\textwidth]{black-knight.png}    };
%\draw[->] (wknight.10) -| (process1);
%\draw[->] (process1) -> (bknight);
%\draw[->] (wknight.350) -| (process2);
%\draw[->] (process2) -> (fknight);
%\draw[->] (process3) -> (result);
%\draw[->] (bknight) |- (process3.190);
%\draw[->] (fknight) |- (process3.170);
%\end{tikzpicture}
\begin{overlayarea}{\textwidth}{6cm}
\only<1>{
\begin{tikzpicture}[remember picture,
  inner/.style={circle,draw=blue!50,fill=blue!20,thick,inner sep=3pt},
  outer/.style={draw=green,fill=green!20,thick,inner ysep=20pt}
  ]
  \node[outer,draw=green] (double) {
    \begin{tikzpicture}
      \node [inner,draw=blue, process,
minimum width=1cm,
minimum height=1cm] (ai)  {flipV};
      \node [inner,draw=blue,process, below=2cm of ai,
minimum width=1cm,
minimum height=1cm] (aii) {invert};
      \node[inner sep=0pt, right=of ai] (fknight)    {\includegraphics[width=.1\textwidth]{flop-knight.png}};
      \node[inner sep=0pt, right=of aii] (bknight)    {\includegraphics[width=.1\textwidth]{black-knight.png}};
      \node [inner,draw=blue, process, below right=.45cm and 1cm of fknight,
minimum width=1cm,
minimum height=1cm] (beside)  {beside};
      \draw[->,thick] (aii) -- (bknight);
      \draw[->,thick] (ai) -- (fknight);
      \draw[->,thick] (bknight) |- (beside.190);
      \draw[->,thick] (fknight) |- (beside.170);
    \end{tikzpicture}
  };
  \node[inner sep=0pt, right=of double] (output)
   {
\includegraphics[width=.1\textwidth]{flop-knight.png}
    \includegraphics[width=.1\textwidth]{black-knight.png}    };
  \node[inner sep=0pt, left=of double] (input)
   {
\includegraphics[width=.1\textwidth]{knight.png}
};
  \draw[thick,->] (beside) -- (output);
  \draw[->,thick] (input.5) -| (ai);
  \draw[->,thick] (input.355) -| (aii);
  \node at (double.north west) [below right,node distance=0 and 0] {double};

\end{tikzpicture}
}
\only<2>{
\vspace{1cm}
\hfill
\begin{tikzpicture}
\node[inner sep=0pt] (wknight)
    {\includegraphics[width=.1\textwidth]{knight.png}};
\node (process1) [draw, process,
minimum width=1cm,
minimum height=1cm, right=of wknight] {double};
\node[inner sep=0pt, right=of process1] (bknight)
    {\includegraphics[width=.1\textwidth]{flop-knight.png}
    \includegraphics[width=.1\textwidth]{black-knight.png}};
\draw[->] (wknight) -> (process1);
\draw[->] (process1) -> (bknight);
\end{tikzpicture}
\hfill
}
\vfill
\end{overlayarea}
\end{frame}

\begin{frame}{Terminologie}
\begin{block}{Prototipul funcției \hfill
\structure{double} {\color{black}:: Picture -> Picture}}
\begin{itemize}
\item \structure{Numele funcției}
\item Signatura funcției
\end{itemize}
\end{block}
\begin{block}{Definiția funcției \hfill \structure{double} \alert{p} {\color{black}= beside (flipV p) (invert p)}}

\begin{itemize}
\item \structure{numele funcției}
\item \alert{parametrul formal}
\item corpul funcției
\end{itemize}
\end{block}
\begin{block}{Aplicarea funcției \hfill \structure{double} \alert{knight}}
\begin{itemize}
\item \structure{numele funcției}
\item \alert{parametrul actual (argumentul)}
\end{itemize}
\end{block}
\end{frame}
\end{section}

\begin{section}{Liste}

\begin{frame}[fragile]{Operatorii \structure{:} și \structure{++}}{Mod de folosire}
\vspace{-3ex}
\begin{minipage}[t]{.49\columnwidth}
\begin{asciihs}
Prelude> :t (:)
(:) :: a -> [a] -> [a]

Prelude> 1 : [2,3]
[1,2,3]


Prelude> :t "bcd"
"bc" :: [Char]
Prelude> 'a' : "bcd"
"abcd"
\end{asciihs}
\end{minipage}
\begin{minipage}[t]{.49\columnwidth}
\begin{asciihs}
Prelude> :t (++)
(++) :: [a] -> [a] -> [a]

Prelude> [1] ++ [2,3]
[1,2,3]
Prelude> [1,2] ++ [3]
[1,2,3]
Prelude> "a" ++ "bcd"
"abcd"
Prelude> "ab" ++ "cd"
"abcd"
\end{asciihs}
\end{minipage}
\begin{itemize}
\item[:] (\structure{cons}) Construiește o listă nouă având primul argument ca prim element și continuând cu al doilea argument ca restul listei.
\item[++] (\structure{append}) Construiește o listă nouă obținută prin alipirea celor două liste argument
\item[\texttt{\footnotesize [Char]}] Șirurile de caractere (\structure{String}) sunt liste de caractere (\structure{Char})
\end{itemize}

\end{frame}

\begin{frame}[fragile]{Operatorii \structure{:} și \structure{++}}{Erori de începător}
\begin{asciihs}
Prelude> :t (:)
(:) :: a -> [a] -> [a]

Prelude> :t (++)
(++) :: [a] -> [a] -> [a]

\end{asciihs}

\begin{minipage}[t]{.49\columnwidth}
\begin{asciihs}
Prelude> [1,2] : 3
  -- eroare de tipuri
Prelude> 1 ++ [2,3]
  -- eroare de tipuri
Prelude> [1] : [2,3]
  -- eroare de tipuri
\end{asciihs}

\end{minipage}
\begin{minipage}[t]{.49\columnwidth}
\begin{asciihs}
Prelude> "ab" : 'c'
  -- eroare de tipuri
Prelude> 'a' ++ "bc"
  -- eroare de tipuri
Prelude> "a" : "bc"
  -- eroare de tipuri
\end{asciihs}

\end{minipage}
\end{frame}

\begin{frame}{Liste}{Definiție}
\begin{block}
{Observație}
Orice listă poate fi scrisă folosind doar constructorul \structure{(:)} și lista vidă \structure{[]}
\begin{itemize}
\item\ [1,2,3] == 1 : (2 : (3 : [])) == 1 : 2 : 3 : []
\item "abcd" == ['a','b','c','d'] == 'a' : ('b' : ('c' : ('d' : []))) == 'a' : 'b' : 'c' : 'd' : []
\end{itemize}
\end{block}
\vfill
\begin{block}{Definiție recursivă}
O \structure{listă} este
\begin{itemize}
\item \structure{vidă}, notată \structure{[]}; sau
\item \structure{compusă}, notată \structure{x:xs}, dintr-un un element \structure{x} numit \structure{capul listei (head)} și o listă \structure{xs} numită \structure{coada listei (tail)}.
\end{itemize}
\end{block}
\vfill
\end{frame}

\begin{frame}[fragile]{Procesarea listelor}

\begin{minipage}[b]{.49\columnwidth}
\begin{asciihs}
Prelude> null [1,2,3]
False
Prelude> head [1,2,3]
1
Prelude> tail [1,2,3]
[2,3]
Prelude> null [2,3]
False
Prelude> head [2,3]
2
Prelude> tail [2,3]
[3]
\end{asciihs}
\end{minipage}
\begin{minipage}{.49\columnwidth}
\begin{asciihs}
Prelude> null [3]
False
Prelude> head [3]
3
Prelude> tail [3]
[]
Prelude> null []
True
\end{asciihs}
\end{minipage}
\end{frame}

\end{section}

\begin{section}{Transformarea fiecărui element dintr-o listă}
\begin{frame}[fragile]{Problemă și abordare}
\begin{block}{}
Definiți o funcție care pentru o listă de numere întregi dată  ridică la pătrat fiecare element din lista.
\end{block}
\begin{block}{Soluție \structure{descriptivă}}
\begin{asciihs}
squares :: [Int] -> [Int]
squares xs = [ x * x | x <- xs ]
\end{asciihs}
\end{block}
\begin{block}{Soluție \structure{recursivă}}
\begin{asciihs}
squaresRec :: [Int] -> [Int]
squaresRec []     = []
squaresRec (x:xs) = x*x : squaresRec xs
\end{asciihs}
\end{block}
\end{frame}

\begin{frame}[fragile]{Variante recursive}
\begin{block}{Ecuațional (pattern matching)}
\vspace{-2ex}
\begin{asciihs}
squaresRec :: [Int] -> [Int]
squaresRec []     = []
squaresRec (x:xs) = x*x : squaresRec xs
\end{asciihs}
\end{block}
\begin{block}{Condițional (cu operatori de legare)}
\vspace{-2ex}
\begin{asciihs}
squaresCond :: [Int] -> [Int]
squaresCond ys = 
  if null ys then []
  else let
         x  = head ys
         xs = tail ys
       in
         x*x : squaresCond xs 
\end{asciihs}
\end{block}
\end{frame}

\begin{frame}{Recursia în acțiune}
\begin{minipage}{.4\columnwidth}%
squaresRec :: [Int] -> [Int]
\end{minipage}
%
\begin{tabular}{l@{ = }l}
\structure<6>{squaresRec []} & \structure<6>{[]}
\\
\structure<3-5>{squaresRec ({\color<3-5>{green!50!black}x}:{\color<3-5>{brown!50!black}xs})} & \structure<3-5>{{\color<3-5>{green!50!black}x}*{\color<3-5>{green!50!black}x} : squaresRec {\color<3-5>{brown!50!black}xs}}
\end{tabular}

\medskip
\structure<1>{squaresRec \structure<2>{[1,2,3]}}
\onslide<2->

=

\structure<3>{squaresRec \structure<2>{({\color<3>{green!50!black}1} : {\color<3>{brown!50!black}(2 : (3 : []))})}}
\onslide<3->

=\only<3>{\hfill \structure{$\{x\mapsto \textrm{1}, xs \mapsto \textrm{2 : (3 : [])}\}$}}

\structure<3>{{\color<3>{green!50!black}1} * {\color<3>{green!50!black}1} : \structure<4>{squaresRec {\color<3>{brown!50!black}({\color<4>{green!50!black}2} : {\color<4>{brown!50!black}(3 : [])})}}}
\onslide<4->

=\only<4>{\hfill \structure{$\{x\mapsto \textrm{2}, xs \mapsto \textrm{3 : []}\}$}}

1 * 1 : \structure<4>{({\color<4>{green!50!black}2} * {\color<4>{green!50!black}2} : \structure<5>{squaresRec {\color<4>{brown!50!black}({\color<5>{green!50!black}3} : {\color<5>{brown!50!black}[]})}})}
\onslide<5->

=\only<5>{\hfill \structure{$\{x\mapsto \textrm{3}, xs \mapsto \textrm{[]}\}$}}

1 * 1 : (2 * 2 : \structure<5>{( {\color<5>{green!50!black}3} * {\color<5>{green!50!black}3} : \structure<6>{squaresRec {\color<5>{brown!50!black}[]}})})
\onslide<6->

=

\structure<7>{1 * 1} : (\structure<7>{2 * 2} : ( \structure<7>{3 * 3} : \structure<6>{[]}))
\onslide<7->

=

\structure<8>{\structure<7>{1} : (\structure<7>{4} : (\structure<7>{9} : []))}
\onslide<8>
= \structure<8>{[1,4,9]} 
\end{frame}
\end{section}

\begin{section}{Selectarea elementelor dintr-o listă}
\begin{frame}[fragile]{Problemă și abordare}
\begin{block}{}
Definiți o funcție care dată fiind o listă de numere întregi selectează doar elementele impare din listă.
\end{block}
\begin{block}{Soluție \structure{descriptivă}}
\begin{asciihs}
odds :: [Int] -> [Int]
odds xs = [ x | x <- xs, odd x ]
\end{asciihs}
\end{block}
\begin{block}{Soluție \structure{recursivă}}
\begin{asciihs}
oddsRec :: [Int] -> [Int]
oddsRec []                 = []
oddsRec (x:xs) | odd x     = x : oddsRec xs
               | otherwise = oddsRec xs
\end{asciihs}
\end{block}
\end{frame}

\begin{frame}[fragile]{Variante recursive}
\begin{block}{Ecuațional (pattern matching)}
\vspace{-2ex}
\begin{asciihs}
oddsRec :: [Int] -> [Int]
oddsRec []                 = []
oddsRec (x:xs) | odd x     = x : oddsRec xs
               | otherwise = oddsRec xs
\end{asciihs}
\end{block}
\begin{block}{Condițional (cu operatori de legare)}
\vspace{-2ex}
\begin{asciihs}
oddsCond :: [Int] -> [Int]
oddsCond ys = 
  if null ys then []
  else let
         x  = head ys
         xs = tail ys
       in
         if odd x then x : oddsCond xs
         else oddsCond xs 
\end{asciihs}
\end{block}
\end{frame}

\begin{frame}{Recursia în acțiune}
oddsRec :: [Int] -> [Int]
\hfill
\begin{tabular}[t]{lcl@{${}={}$}l@{}}
{\color<6>{blue}oddsRec []}     &        &           & {\color<6>{blue}[]} 
\\
{\color<3-5>{blue}oddsRec ({\color<3-5>{green!50!black}x}:{\color<3-5>{brown!50!black}xs})} & $\mid$ & {\color<3,5>{blue}odd {\color<3,5>{green!50!black}x}}     & {\color<3,5>{blue}{\color<3,5>{green!50!black}x} : oddsRec {\color<3,5>{brown!50!black}xs}}
\\
               & $\mid$ & {\color<4>{blue}otherwise} & {\color<4>{blue}oddsRec {\color<4>{brown!50!black}xs}}
\end{tabular}

\medskip
\structure<1>{oddsRec \structure<2>{[1,2,3]}}
\onslide<2->

=

\structure<3>{oddsRec \structure<2>{({\color<3>{green!50!black}1} : {\color<3>{brown!50!black}(2 : (3 : []))})}}
\onslide<3->

=\only<3>{\hfill \structure{$\{x\mapsto \textrm{1}, xs \mapsto \textrm{2 : (3 : [])}\}$; odd 1 = True}}

\structure<3>{{\color<3>{green!50!black}1} : \structure<4>{oddsRec {\color<3>{brown!50!black}({\color<4>{green!50!black}2} : {\color<4>{brown!50!black}(3 : [])})}}}
\onslide<4->

=\only<4>{\hfill \structure{$\{x\mapsto \textrm{2}, xs \mapsto \textrm{3 : []}\}$; odd 2 = False}}

1 : \structure<4>{ \structure<5>{oddsRec {\color<4>{brown!50!black}({\color<5>{green!50!black}3} : {\color<5>{brown!50!black}[]})}}}
\onslide<5->

=\only<5>{\hfill \structure{$\{x\mapsto \textrm{3}, xs \mapsto \textrm{[]}\}$; odd 3 = True }}

1 : \structure<5>{( {\color<5>{green!50!black}3} : \structure<6>{oddsRec {\color<5>{brown!50!black}[]}})}
\onslide<6->

=

\structure<7>{1 : ( 3 : \structure<6>{[]})}
\onslide<7->
=
\structure<7>{[1,3]} 
\end{frame}

\end{section}

\begin{section}{Agregarea elementelor dintr-o listă}
\begin{frame}[fragile]{Problemă și abordare}
\begin{block}{}
Definiți o funcție care dată fiind o listă de numere întregi calculează suma elementelor din listă.
\end{block}
\begin{block}{Soluție \structure{recursivă}}
\begin{asciihs}
suma :: [Int] -> Int
suma []     = 0
suma (x:xs) = x + suma xs
\end{asciihs}
\end{block}
\end{frame}

\begin{frame}{Recursia în acțiune}
suma :: [Int] -> Int
\hfill
\begin{tabular}[t]{l@{${}={}$}l@{}}
{\color<6>{blue}suma []}             & {\color<6>{blue}0} 
\\
{\color<3-5>{blue}suma ({\color<3-5>{green!50!black}x}:{\color<3-5>{brown!50!black}xs})} & {\color<3-5>{blue}{\color<3-5>{green!50!black}x} + suma {\color<3,5>{brown!50!black}xs}}
\end{tabular}

\medskip
\structure<1>{suma \structure<2>{[1,2,3]}}
\onslide<2->

=

\structure<3>{suma \structure<2>{({\color<3>{green!50!black}1} : {\color<3>{brown!50!black}(2 : (3 : []))})}}
\onslide<3->

=\only<3>{\hfill \structure{$\{x\mapsto \textrm{1}, xs \mapsto \textrm{2 : (3 : [])}\}$}}

\structure<3>{{\color<3>{green!50!black}1} + \structure<4>{suma {\color<3>{brown!50!black}({\color<4>{green!50!black}2} : {\color<4>{brown!50!black}(3 : [])})}}}
\onslide<4->

=\only<4>{\hfill \structure{$\{x\mapsto \textrm{2}, xs \mapsto \textrm{3 : []}\}$}}

1 + \structure<4>{({\color<4>{green!50!black}2} + \structure<5>{suma {\color<4>{brown!50!black}({\color<5>{green!50!black}3} : {\color<5>{brown!50!black}[]})}})}
\onslide<5->

=\only<5>{\hfill \structure{$\{x\mapsto \textrm{3}, xs \mapsto \textrm{[]}\}$}}

1 + (2 + \structure<5>{( {\color<5>{green!50!black}3} + \structure<6>{suma {\color<5>{brown!50!black}[]}})})
\onslide<6->

=

\structure<7>{1 + (2 + ( 3 + \structure<6>{0}))}
\onslide<7->
=
\structure<7>{6} 
\end{frame}

\begin{frame}[fragile]{Problemă și abordare}
\begin{block}{}
Definiți o funcție care dată fiind o listă de numere întregi calculează produsul elementelor din listă.
\end{block}
\begin{block}{Soluție \structure{recursivă}}
\begin{asciihs}
produs :: [Int] -> Int
produs []     = 1
produs (x:xs) = x * produs xs
\end{asciihs}
\end{block}
\end{frame}

\begin{frame}{Recursia în acțiune}
produs :: [Int] -> Int
\hfill
\begin{tabular}[t]{l@{${}={}$}l@{}}
{\color<6>{blue}produs []}             & {\color<6>{blue}1} 
\\
{\color<3-5>{blue}produs ({\color<3-5>{green!50!black}x}:{\color<3-5>{brown!50!black}xs})} & {\color<3-5>{blue}{\color<3-5>{green!50!black}x} * produs {\color<3,5>{brown!50!black}xs}}
\end{tabular}

\medskip
\structure<1>{produs \structure<2>{[1,2,3]}}
\onslide<2->

=

\structure<3>{produs \structure<2>{({\color<3>{green!50!black}1} : {\color<3>{brown!50!black}(2 : (3 : []))})}}
\onslide<3->

=\only<3>{\hfill \structure{$\{x\mapsto \textrm{1}, xs \mapsto \textrm{2 : (3 : [])}\}$}}

\structure<3>{{\color<3>{green!50!black}1} * \structure<4>{produs {\color<3>{brown!50!black}({\color<4>{green!50!black}2} : {\color<4>{brown!50!black}(3 : [])})}}}
\onslide<4->

=\only<4>{\hfill \structure{$\{x\mapsto \textrm{2}, xs \mapsto \textrm{3 : []}\}$}}

1 * \structure<4>{({\color<4>{green!50!black}2} * \structure<5>{produs {\color<4>{brown!50!black}({\color<5>{green!50!black}3} : {\color<5>{brown!50!black}[]})}})}
\onslide<5->

=\only<5>{\hfill \structure{$\{x\mapsto \textrm{3}, xs \mapsto \textrm{[]}\}$}}

1 * (2 * \structure<5>{( {\color<5>{green!50!black}3} * \structure<6>{produs {\color<5>{brown!50!black}[]}})})
\onslide<6->

=

\structure<7>{1 * (2 * ( 3 * \structure<6>{1}))}
\onslide<7->
=
\structure<7>{6} 
\end{frame}

\end{section}

\begin{section}{Mapare, filtrare și agregare deodată}
\end{section}
\begin{frame}[fragile]{Problemă și abordare}
\begin{block}{}
Definiți o funcție care dată fiind o listă de numere întregi calculează suma pătratelor elementelor impare din listă.
\end{block}
\begin{block}{Soluție \structure{descriptivă}}
\begin{asciihs}
sumSqOdd :: [Int] -> Int
sumSqOdd xs = sum [ x * x | x <- xs, odd x ]
\end{asciihs}
\end{block}
\begin{block}{Soluție \structure{recursivă}}
\begin{asciihs}
sumSqOddRec :: [Int] -> Int
sumSqOddRec []                 = 0
sumSqOddRec (x:xs) | odd x     = x * x + sumSqOddRec xs
                   | otherwise = sumSqOddRec xs
\end{asciihs}
\end{block}
\structure{Soluție \structure{combinatorială}}
\hfill \lstinline$sumSqOdd = sum . squares . odds$

\end{frame}


\begin{frame}{Recursia în acțiune}
oddsRec :: [Int] -> [Int]

\begin{tabular}[t]{@{}lcl@{${}={}$}l@{}}
{\color<6>{blue}sumSqOddRec []}     &        &           & {\color<6>{blue}0} 
\\
{\color<3-5>{blue}sumSqOddRec ({\color<3-5>{green!50!black}x}:{\color<3-5>{brown!50!black}xs})} & $\mid$ & {\color<3,5>{blue}odd {\color<3,5>{green!50!black}x}}     & {\color<3,5>{blue}{\color<3,5>{green!50!black}x*x} + sumSqOddRec {\color<3,5>{brown!50!black}xs}}
\\
               & $\mid$ & {\color<4>{blue}otherwise} & {\color<4>{blue}sumSqOddRec {\color<4>{brown!50!black}xs}}
\end{tabular}

\medskip
\structure<1>{sumSqOddRec \structure<2>{[1,2,3]}}
\onslide<2->
=

\structure<3>{sumSqOddRec \structure<2>{({\color<3>{green!50!black}1} : {\color<3>{brown!50!black}(2 : (3 : []))})}}
\onslide<3->

=\only<3>{\hfill \structure{$\{x\mapsto \textrm{1}, xs \mapsto \textrm{2 : (3 : [])}\}$; odd 1 = True}}

\structure<3>{{\color<3>{green!50!black}1} * {\color<3>{green!50!black}1} + \structure<4>{sumSqOddRec {\color<3>{brown!50!black}({\color<4>{green!50!black}2} : {\color<4>{brown!50!black}(3 : [])})}}}
\onslide<4->

=\only<4>{\hfill \structure{$\{x\mapsto \textrm{2}, xs \mapsto \textrm{3 : []}\}$; odd 2 = False}}

1 * 1 + \structure<4>{ \structure<5>{sumSqOddRec {\color<4>{brown!50!black}({\color<5>{green!50!black}3} : {\color<5>{brown!50!black}[]})}}}
\onslide<5->

=\only<5>{\hfill \structure{$\{x\mapsto \textrm{3}, xs \mapsto \textrm{[]}\}$; odd 3 = True }}

1 * 1 + \structure<5>{( {\color<5>{green!50!black}3} * {\color<5>{green!50!black}3} + \structure<6>{sumSqOddRec {\color<5>{brown!50!black}[]}})}
\onslide<6->

=

\structure<7>{1 * 1 + ( 3 * 3 + \structure<6>{0})}
\onslide<7->
=
\structure<7>{10} 
\end{frame}

\end{document}



