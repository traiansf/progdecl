\documentclass[xcolor=pdftex,romanian,colorlinks]{beamer}

\usepackage[export]{adjustbox}
\usepackage{../tslides}
\usepackage[all]{xy}
\lstset{language=Haskell}
\lstset{escapeinside={(*@}{@*)}}
\PrerenderUnicode{ăĂîÎȘșȚțâÂ}

\AtBeginSection[]{
  \begin{frame}
  \vfill
  \centering
  \begin{beamercolorbox}[sep=8pt,center,shadow=true,rounded=true]{title}
    \usebeamerfont{title}\insertsectionhead\par%
  \end{beamercolorbox}
  \vfill
  \end{frame}
}


\title[PF---Funcții și liste]{Programare Funcțională}
\subtitle{Funcții și liste în Haskell}
\date{}
\begin{document}
\begin{frame}
  \titlepage
\end{frame}

\begin{frame}
\frametitle{}
\tableofcontents
\end{frame}


\begin{section}{Funcții}

\begin{frame}{Funcții în Haskell. Terminologie}
\begin{block}{Prototipul funcției \hfill
{\color{black}{\bf double} {:: Integer -> Integer}}}
\begin{itemize}
\item {\bf numele funcției}
\item signatura funcției
\end{itemize}
\end{block}
\begin{block}{Definiția funcției \hfill {\color{black}{\bf double}} \alert{elem} {\color{black}= elem + elem}}

\begin{itemize}
\item {\bf numele funcției}
\item \alert{parametrul formal}
\item corpul funcției
\end{itemize}
\end{block}
\begin{block}{Aplicarea funcției \hfill {\color{black}{\bf double}} \alert{5}}
\begin{itemize}
\item {\bf numele funcției}
\item \alert{parametrul actual (argumentul)}
\end{itemize}
\end{block}
\end{frame}

\begin{frame}{Exemplu: adunarea a doi \^{\i}ntregi}
\begin{block}{Prototipul funcției \hfill
{\color{black}{\bf add} {:: Integer -> Integer  -> Integer}}}
\begin{itemize}
\item {\bf numele funcției}
\item signatura funcției
\end{itemize}
\end{block}
\begin{block}{Definiția funcției \hfill {\color{black}{\bf add}} \alert{elem1} \alert{elem2} {\color{black}= elem1 + elem2}}

\begin{itemize}
\item {\bf numele funcției}
\item \alert{parametrii formali}
\item corpul funcției
\end{itemize}
\end{block}
\begin{block}{Aplicarea funcției \hfill {\color{black}{\bf add}} \alert{3} \alert{7}}
\begin{itemize}
\item {\bf numele funcției}
\item \alert{argumentele}
\end{itemize}
\end{block}
\end{frame}

\begin{frame}{Exemplu: funcție cu {\bf un} argument de tip tuplu }
\begin{block}{Prototipul funcției \hfill
{\color{black}{\bf dist} {:: (Integer, Integer)  -> Integer}}}
\begin{itemize}
\item {\bf numele funcției}
\item signatura funcției
\end{itemize}
\end{block}
\begin{block}{Definiția funcției \hfill {\color{black}{\bf dist}} \alert{(elem1, elem2)} {\color{black}= abs (elem1 - elem2)}}

\begin{itemize}
\item {\bf numele funcției}
\item \alert{parametrul  formal}
\item corpul funcției
\end{itemize}
\end{block}
\begin{block}{Aplicarea funcției \hfill {\color{black}{\bf dist}} \alert{(2, 5)}}
\begin{itemize}
\item {\bf numele funcției}
\item \alert{argumentul}
\end{itemize}
\end{block}
\end{frame}


%\begin{frame}{Func\ts ii \^{\i}n matematic\u a}
%
%\begin{itemize}
%
%\item Fie $f:A\times B\to C$ o func\ts ie. \^{I}n mod uzual scriem
%
%$f(x,y)=z$  unde $x\in A$, $y\in B$ \sh i $z\in C$.
%\bigskip
%
%\item Pentru $x\in A$ (arbitrar, fixat) definim
%
%\medskip
%\structure{$f_x:B\to C$,  $f_x(y)=z$} dac\u a \sh i numai dac\u a $f(x,y)=z$.
%\bigskip
%
%Func\ts ia $f_x$ se ob\ts ine prin \structure{aplicarea par\ts ial\u a} a func\ts iei $f$.
%\bigskip
%\pause
%
%In mod similar definim {\it aplicarea par\ts ial\u a} pentru orice $y\in B$
%
%\medskip
%
%$f^y:A\to C$,  $f^y(x)=z$ dac\u a \sh i numai dac\u a $f(x,y)=z$.
%
%
%
%
%\end{itemize}
%\end{frame}
%
%
%
%\begin{frame}{Func\ts ii \^{\i}n matematic\u a}
%\begin{block}{Exemplu}
%
%$A=${Int}, $B=C=${String}
%
%$f (x,y)= \left\{\begin{array}{ll}
%z, &  |y|>=x, |z|=x, y =zw\\
%y, &  0<|y|<x\\
%"", & x<=0
% \end{array}\right.$
%
%\end{block}
%
%\begin{itemize}
%\item Fie  $x\in$ {Int} arbitrar, fixat. Atunci
%$f_x:${String}$\to${String} \sh i
%
% - dac\u a $x<=0$, atunci $f_x(y)=""$ oricare $y$
%
% - dac\u a $x>0$ atunci
%$f_x(y)= \left\{\begin{array}{ll}
%z, &  |y|>=x, |z|=x, y =zw\\
%y, &  0<|y|<x
%
% \end{array}\right.$ \pause
%
%\item  Fie  $y\in${String} arbitrar, fixat. Atunci $f^y:${Int}$\to${String} \sh i
%
%$f^y(x)= \left\{\begin{array}{ll}
%z, &  |y|>=x, |z|=x, y =zw\\
%y, &  0<|y|<x\\
%"", & x<=0
% \end{array}\right.$
%\end{itemize}
%\end{frame}
%
%
%\begin{frame}[fragile]{Func\ts ii \^{\i}n matematic\u a}
%
%\begin{itemize}
%\item Fie $f:A\times B\to C$ o func\ts ie. \^{I}n mod uzual scriem
%
%$f(x,y)=z$ unde  $x\in A$, $y\in B$ \sh i $z\in C$.
%\medskip
%
%\item Pentru $x\in A$ (arbitrar, fixat) definim
%
%$f_x:B\to C$,  $f_x(y)=z$ dac\u a \sh i numai dac\u a $f(x,y)=z$.
%
%\medskip\pause
%
%\item Dac\u a not\u am \structure{$B\to C \stackrel{not}{=}\{h:B\to C\mid h\mbox{ func\ts ie}\}$}
%
%observ\u am c\u a \structure{$f_x\in B\to C $} pentru orice $x\in A$.
%
%\medskip\pause
%\item Asociem lui $f$ func\ts ia
%\medskip
%
%\structure{$cf:A\to (B\to C)$, $\,\, cf(x)=f_x$}
%\medskip
%
%
%Observ\u am c\u a pentru fiecare element $x\in A$, func\ts ia $cf$ \^{\i}ntoarce ca rezultat func\ts ia $f_x\in B\to C$,
%adic\u a
%
%\medskip
%\structure{$cf(x)(y)=z$ dac\u a \sh i numai dac\u a $f(x,y)=z$}
%
%\end{itemize}
%\pause
%
%\begin{block}{Forma {\bf curry}}
%Vom spune c\u a func\ts ia \structure{$cf$} este {\it forma curry} a func\ts iei $f$.
%\end{block}
%
%\end{frame}
%
%\begin{frame}[fragile]{De la matematic\u a la Haskell}
%
%Func\ts ia  $f:$ Int $\times$ String $\to$ String
%
%$f (x,y)= \left\{\begin{array}{ll}
%z, &  |y|>=x, |z|=x, y =zw\\
%y, &  0<|y|<x\\
%"", & x<=0
% \end{array}\right.$
%
%poate fi definit\u a \^{\i}n Haskell astfel:
%
%\begin{asciihs}
%f :: (Int, String) -> String
%f (n,s) = take n s
%\end{asciihs}
%\pause
%Observ\u am c\u a:
%
%\begin{asciihs}
%Prelude> cf = curry f
%Prelude> :t cf
%cf :: Int -> String -> String
%Prelude> f(1,"abc")
%"a"
%Prelude> cf 1 "abc"
%"a"
%\end{asciihs}
%
%\end{frame}
%
%
%
%
%
%\begin{frame}{Currying}
%
%\begin{block}{}
%\structure{"Currying"} este procedeul prin care o func\ts ie cu mai multe argumente este transformat\u a \^{\i}ntr-o func\ts ie care are un singur argument \sh i \^{\i}ntoarce o alt\u a func\ts ie.
%\end{block}
%\bigskip
%
%
%\begin{itemize}
%\item In Haskell toate funcțiile sunt forma {\bf curry}, deci au un singur argument.
%\medskip
%
%\item Operatorul  $\to$ pe tipuri este asociativ la dreapta, adic\u a tipul
%\structure{$a_1 \to a_2 \to  \cdots  \to  a_n$}   \^{\i}l g\^ andim ca
%\structure{$a_1\to (a_2\to \cdots (a_{n-1}\to a_n)\cdots )$}.
%
%\medskip
%
%\item  Aplicarea func\ts iilor este  asociativ\u a la st\^ anga, adic\u a
%
%expresia \structure{$f \,\, x_1\cdots    x_n$} o  g\^ andim ca
%\structure{$ (\cdots ((f\,\, x_1) \, x_2)\cdots x_n)$}.
%\end{itemize}
%\end{frame}
%
%
%\begin{frame}[fragile]{Func\ts ii \sh i mul\ts imi}
%\begin{block}{Teorem\u a}
%Mul\ts imile $(A\times B)\to C$ \sh i $A\to (B\to C)$ sunt echipotente.
%\end{block}
%\medskip\pause
%
%
%\begin{block}{Observa\ts ie}
%Func\ts iile \structure{curry} \sh i \structure{uncurry} din Haskell stabilesc bijec\ts ia din teorem\u a:
%\begin{asciihs}
%Prelude> :t curry
%curry :: ((a, b) -> c) -> a -> b -> c
%Prelude> :t uncurry
%uncurry :: (a -> b -> c) -> (a, b) -> c
%\end{asciihs}
%\end{block}
%
%\end{frame}
%

\begin{frame}[fragile]{Tipuri de funcții}

Fie {foo} o funcție cu următorul tip


\begin{asciihs}
foo :: a -> b -> [a] -> [b]
\end{asciihs}


\begin{itemize}

\item are trei argumente, de tipuri a, b și [a]
\item întoarce un rezultat de tip [b]
\end{itemize}
\pause\bigskip

Schimbăm signatura funcției astfel:

\begin{asciihs}
ffoo :: (a -> b) -> [a] -> [b]
\end{asciihs}



\begin{itemize}

\item are două argumente, de tipuri (a -> b) și [a],

adică o funcție de la a la b și o listă de elemente de tip a
\item întoarce un rezultat de tip [b]
\end{itemize}
\pause
\begin{asciihs}
Prelude> :t map
map :: (a -> b) -> [a] -> [b]

\end{asciihs}
\end{frame}




\begin{subsection}{Șabloane}


\begin{frame}[fragile]{Definirea funcțiilor folosind {\bf if} }

\begin{itemize}
\item \structure{analiza cazurilor folosind expresia "if"}



\begin{asciihs}
semn : Integer -> Integer
semn n = if n < 0 then (-1)
         else if n=0 then 0
              else 1
\end{asciihs}

\item \structure{ defini\ts ie recursiv\u a \^{\i}n care analiza cazurilor folose\sh te expresia "if"}
\begin{asciihs}
fact :: Integer -> Integer
fact n = if n == 0 then 1
         else n * fact(n-1)
\end{asciihs}

\end{itemize}
\end{frame}

\begin{frame}[fragile]{Definirea funcțiilor folosind {\bf g\u arzi}}


Func\ts ia $semn$ o putem defini astfel

\medskip

 $semn \,\, n=\left\{\begin{array}{ll}
 -1, & \mbox{\bf dac\u a } {\bf n < 0}\\
 0, & \mbox{\bf dac\u a } {\bf n = 0}\\

 1, & \mbox{\bf altfel}
 \end{array}\right.$
 \medskip

\^ In Haskell, condi\ts iile devin \structure{g\u arzi}:

\medskip

\begin{asciihs}
semn n
   (*@ {\bf | n < 0} @*)     = -1
   (*@ {\bf | n = 0} @*)     =  0
   (*@ {\bf | otherwise @*) = 1
\end{asciihs}

\end{frame}

\begin{frame}[fragile]{Definirea funcțiilor folosind {\bf g\u arzi}}


Func\ts ia $fact$ o putem defini astfel

\medskip

 $fact \,\, n=\left\{\begin{array}{ll}
 1, & \mbox{\bf dac\u a } {\bf n=0}\\
 n* fact(n-1), & \mbox{\bf altfel}
 \end{array}\right.$

 \medskip

\^ In Haskell, condi\ts iile devin \structure{g\u arzi}:

\medskip

\begin{asciihs}
fact n
   (*@ {\bf | n == 0} @*)    = 1
   (*@ {\bf | otherwise @*) = n * fact(n-1)
\end{asciihs}




\end{frame}



\begin{frame}[fragile]{Definirea funcțiilor folosind \sh abloane \sh i ecua\ts ii}


\begin{asciihs}
semn :: Integer -> Integer
semn 0 = 0
semn x
  | x > 0     = 1
  | otherwise = -1
\end{asciihs}



\begin{asciihs}
fact :: Integer -> Integer
fact 0 = 1
fact n = n * fact(n-1)
\end{asciihs}

\begin{itemize}
\item variabilele \sh i valorile din partea st\^ ang\u a a semnului = sunt {\it \sh abloane};
\item c\^ and func\ts ia este aplelat\u a se \^{\i}ncearc\u a potrivarea parametrilor actuali cu \sh abloanele, ecua\ts iile fiind \^{\i}ncercate {\it  \^{\i}n ordinea scrierii};
\item \^{\i}n defini\ts ia factorialului, \texttt{$0$} \sh i n sunt \sh abloane: \texttt{$0$} se va potrivi numai cu el \^{\i}nsu\sh i, iar n se va potrivi cu orice valoare de tip
\texttt{Integer}.
\end{itemize}

\end{frame}


\begin{frame}[fragile]{Definirea funcțiilor folosind \sh abloane \sh i ecua\ts ii}


\begin{itemize}
\item \structure{\^{\i}n Haskell, ordinea ecua\ts iilor este  important\u a}
\end{itemize}


S\u a presupunem c\u a schimb\u am   ordinii ecua\ts iilor din defini\ts ia factorialului:

\begin{asciihs}
fact :: Integer -> Integer
fact n = n * fact(n-1)
fact 0 = 1
\end{asciihs}

Ce se \^{\i}nt\^{a}mpl\u a?
\medskip\pause

Deoarece n este un pattern care se potrive\sh te cu orice valoare, inclusiv cu $0$,  orice apel al func\ts iei va alege prima ecua\ts ie.  Astfel, func\ts ia \textcolor{red}{nu} \^{\i}\sh i va \^{\i}ncheia execu\ts ia pentru valori pozitive.
\end{frame}

\begin{frame}[fragile]{Definirea funcțiilor folosind \sh abloane \sh i ecua\ts ii}

Tipul Bool este definit \^{\i}n Haskell astfel:
\begin{asciihs}
data Bool = True | False
\end{asciihs}

Putem defini opera\ts ia $||$ astfel

\begin{asciihs}
(||) :: Bool -> Bool -> Bool

False || x = x
True  || _ = True
\end{asciihs}


\^{I}n acest exemplu \sh abloanele sunt {\bf $\_$}, {\bf True} \sh i {\bf False}.
\medskip

Observ\u am c\u a {\bf True} \sh i {\bf False} sunt constructori de date \sh i se vor potrivi numai cu ei \^{\i}n\sh i\sh i.
\medskip

 Șablonul {\bf $\_$} se nume\sh te {\it wild-card pattern}; el se potrive\sh te cu orice valoare.


\end{frame}


\begin{frame}[fragile]{\Sh abloane pentru tupluri}

  Observați că (,) este constructorul pentru perechi.
  
  \begin{asciihs}
  (u,v)=('a',[(1,'a'),(2,'b')]) -- u='a',
                                -- v=[(1,'a'),(2,'b')]
  \end{asciihs}
  \begin{itemize}
  \pause
  
  \item Definitii folosind șabloane
  \begin{asciihs}
  selectie :: Integer -> String -> String
  \end{asciihs}
  \vspace{-2ex}
  \begin{columns}[t]
    \begin{column}{.45\columnwidth}
  \begin{asciihs}
  -- case...of
  selectie x s =
      case (x,s) of
          (0,_) -> s
          (1, z:zs) -> zs
          (1, []) -> []
          _ -> (s ++ s)
  \end{asciihs}
    \end{column}
    \begin{column}{.45\columnwidth}
  \begin{asciihs}
  -- stil ecuational
  selectie 0 s = s
  selectie 1 (_:s) = s
  selectie 1 "" = ""
  selectie _ s = s + s
  \end{asciihs}
    \end{column}
  \end{columns}
  
  \end{itemize}
  \end{frame}
  
\end{subsection}

\end{section}


%-----------------------------------------------------
\begin{section}{Liste}
  \begin{frame}{Liste}{Definiție}
  \begin{block}
  {Observație}
  Orice listă poate fi scrisă folosind doar constructorul \structure{(:)} și lista vidă \structure{[]}
  \begin{itemize}
  \item\ [1,2,3] == 1 : (2 : (3 : [])) == 1 : 2 : 3 : []
  \item "abcd" == ['a','b','c','d'] == 'a' : ('b' : ('c' : ('d' : []))) == 'a' : 'b' : 'c' : 'd' : []
  \end{itemize}
  \end{block}
  \vfill
  \begin{block}{Definiție recursivă}
  O \structure{listă} este
  \begin{itemize}
  \item \structure{vidă}, notată \structure{[]}; sau
  \item \structure{compusă}, notată \structure{x:xs}, dintr-un un element \structure{x} numit \structure{capul listei (head)} și o listă \structure{xs} numită \structure{coada listei (tail)}.
  \end{itemize}
  \end{block}
  \vfill
  \end{frame}
  
  \begin{frame}[fragile]{Definirea listelor. Operații}
  \begin{block}{ Intervale și progresii}
  \begin{asciihs}
  interval = ['c'..'e']       -- ['c','d','e']
  progresie = [20,17..1]      -- [20,17,14,11,8,5,2]
  progresie' = [2.0,2.5..4.0] -- [2.0,2.5,3.0,3.5,4.0]
  \end{asciihs}
  \end{block}
  \pause
  \begin{block}{Operații}
  \begin{asciihs}
  Prelude> [1,2,3] !! 2
  3
  Prelude> "abcd" !! 0
  'a'
  Prelude> [1,2] ++ [3]
  [1,2,3]
  Prelude> import Data.List
  \end{asciihs}
  \end{block}
  \end{frame}
  
  \begin{frame}[fragile]{Definiția prin selecție $\{x\mid P(x)\}$}
  \structure{[E(x)| x <- [x1,…,xn], P(x)]}
  
  \begin{asciihs}
  Prelude> xs = [0..10]
  Prelude> [x | x <- xs, even x](*@\pause@*)
  [0,2,4,6,8,10]
  
  Prelude> xs = [0..6]
  Prelude> [(x,y) | x <- xs, y <- xs,  x + y == 10](*@\pause@*)
  [(4,6),(5,5),(6,4)]
  \end{asciihs}
  Folosirea lui \structure{\tt{let}} pentru declarații locale:
  \begin{asciihs}
  Prelude> [(i,j) | i <- [1..2], (*@\color{blue}{let}@*) k = 2 * i, j <- [1..k]](*@\pause@*)
  [(1,1),(1,2),(2,1),(2,2),(2,3),(2,4)]
  
  Prelude> xs = ['A'..'Z']
  Prelude> [x | (i,x) <- [1..] `zip` xs, even i](*@\pause@*)
  "BDFHJLNPRTVXZ"
  \end{asciihs}
  
  \end{frame}
  
  
  \begin{frame}[fragile]{zip xs ys}
  \begin{asciihs}
  Prelude> xs = ['A'..'Z']
  Prelude> [x | (i,x) <- [1..] `zip` xs, even i](*@\pause@*)
  "BDFHJLNPRTVXZ"
  
  Prelude> :t zip
  zip :: [a] -> [b] -> [(a, b)]
  
  Prelude> ys = ['A'..'E']
  Prelude> zip [1..]  ys
  [(1,'A'),(2,'B'),(3,'C'),(4,'D'),(5,'E')]
  \end{asciihs}
  
  \medskip\pause
  
  \structure{Observați diferența!}
  
  \begin{asciihs}
  Prelude> zip [1..3] ['A'..'D']
  [(1,'A'),(2,'B'),(3,'C')]
  
  Prelude> [(x,y) | x <- [1..3], y <- ['A'..'D']]
  [(1,'A'),(1,'B'),(1,'C'),(1,'D'),(2,'A'),(2,'B'),(2,'C'),(2,'D'),(3,'A'),(3,'B'),(3,'C'),(3,'D')]
  \end{asciihs}
  
  \end{frame}
  
  \begin{frame}[fragile]{Lenevire (Lazyness)}
  
   Argumentele sunt evaluate doar când e necesar și doar cât e necesar
  
  
  \begin{asciihs}
  Prelude> head[]
  *** Exception: Prelude.head: empty list
  Prelude> x = head []
  Prelude> f a = 5
  Prelude> f x
  5
  Prelude> [1,head [],3] !! 0
  1
  Prelude> [head [],3] !! 1
  *** Exception: Prelude.head: empty list
  \end{asciihs}
  
  \end{frame}
  
  \begin{frame}[fragile]{Liste infinite}
  
  
  \begin{block}{}
  
  Drept consecință a \structure{evaluării leneșe}, se pot defini liste infinite (fluxuri de date)
  \end{block}
  
  \begin{asciihs}
  Prelude> natural = [0 ..]
  Prelude> take 5 natural
  [0,1,2,3,4]
  \end{asciihs}
  \pause
  \begin{asciihs}
  Prelude> evenNat = [0, 2 ..] -- progresie infinita
  Prelude> take 7 evenNat
  [0,2,4,6,8,10,12]
  \end{asciihs}
  \pause
  \begin{asciihs}
  Prelude> ones = [1,1..]
  Prelude> zeros = [0,0..]
  Prelude> both = zip ones zeros
  Prelude> take 5 both
  [(1,0),(1,0),(1,0),(1,0),(1,0)]
  \end{asciihs}
  
  
  \end{frame}
  
  
\begin{frame}[fragile]{\Sh abloane (patterns) pentru liste}

  Listele sunt construite folosind constructorii (:) \sh i []
  
  \begin{asciihs}
  [1,2,3] == 1:[2,3]  -- == 1:2:[3] == 1:2:3:[]
  \end{asciihs}
  
  Observa\ts i:
  
  \begin{asciihs}
  Prelude> x:y = [1,2,3]
  Prelude> x
  1
  Prelude> y
  [2,3]
  \end{asciihs}
  
  Ce s-a \^{\i}nt\^amplat?
  \begin{itemize}
  
  \item \structure{x:y} este un \sh ablon pentru liste
  \item potrivirea dintre \structure{x:y} \sh i \structure{[1,2,3]}
   a avut ca efect:
  \begin{itemize}
  \item "deconstruc\ts ia" valorii [1,2,3]  \^{\i}n 1:[2,3]
  \item  legarea lui x la 1 \sh i a lui y la [2,3]
  \end{itemize}
   \end{itemize}
  \end{frame}
  
  
  \begin{frame}[fragile]{\Sh abloane (patterns) pentru liste}
  
   Defini\ts ii folosind \sh abloane
  
  \begin{asciihs}
  reverse [] = []
  reverse (x:xs) = (reverse xs) ++ [x]
  \end{asciihs}
  
  \begin{itemize}
  \item \structure{x:xs} se potrive\sh te cu liste nevide
  \end{itemize}
  \pause
  
  \begin{block}{Aten\ts ie!} \structure{\Sh abloanele sunt definite folosind constructori.} De exemplu, opera\ts ia de concatenare pe liste este
  (++) :: [a]-> [a] -> [a] dar
  
  [x] ++ [1] = [2,1] \textcolor{red}{nu} va avea ca efect legarea lui x la 2;
  
  \^{\i}ncerc\^ and s\u a evalu\u am x vom ob\ts ine un mesaj de eroare:
  \begin{asciihs}
  Prelude> [x] ++ [1] = [2,1]
  Prelude> x
  (*@\color{red}{error: ...}@*)
  \end{asciihs}
  
  \end{block}
  
  
  \end{frame}
  
  
   

\begin{frame}[fragile]{Șabloanele sunt liniare}

\^{I}n Haskell \sh abloanele sunt \structure{liniare}, adic\u a o variabil\u a apare cel mult odat\u a.

\Sh abloane  \^{\i}n care o variabil\u a apare de mai multe ori provoac\u a mesaje de eroare

\begin{asciihs}
x:x:[1] = [2,2,1]

ttail (x:x:t) = t

foo x x = x^2
(*@\pause\color{red}{error: Conflicting definitions for x}@*)
\end{asciihs}

 \pause

O solu\ts ie este folosirea g\u arzilor:

\begin{asciihs}
ttail (x:y:t) | (x==y) = t
               | otherwise = ...

foo x y | (x == y) = x^2
        | otherwise = ...
\end{asciihs}
\end{frame}


\end{section}


 \begin{frame}
  \vfill
  \centering
  \begin{beamercolorbox}[sep=8pt,center,shadow=true,rounded=true]{title}
    \usebeamerfont{title} Pe săptămâna viitoare! \par%
  \end{beamercolorbox}
  \vfill
  \end{frame}


\end{document}


