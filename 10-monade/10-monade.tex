\documentclass[xcolor=pdftex,romanian,colorlinks]{beamer}

\usepackage[export]{adjustbox}
\usepackage{../tslides}
\usepackage[all]{xy}
\usepackage{pgfplots}
\usepackage{flowchart}
\usetikzlibrary{arrows,positioning,calc}
\lstset{language=Haskell}
\lstset{escapeinside={(*@}{@*)}}

\AtBeginSection[]{
  \begin{frame}
  \vfill
  \centering
  \begin{beamercolorbox}[sep=8pt,center,shadow=true,rounded=true]{title}
    \usebeamerfont{title}\insertsectionhead\par%
  \end{beamercolorbox}
  \vfill
  \end{frame}
}


\title[PD---Monade]{Programare declarativă\thanks{bazat pe cursul \emph{Informatics 1: Functional Programming} de la \emph{University of Edinburgh}}}

\subtitle{Efecte laterale --- Monade}

\begin{document}
\begin{frame}
  \titlepage
\end{frame}

\section{Efecte laterale}

\begin{frame}[fragile]{Logging în C}
\begin{asciic}
#include <iostream>
#include <sstream>
using namespace std;
ostringstream log;
int increment(int x) {
  log << "Called increment with argument " << x << endl;
  return x + 1;
}

int main() {
  int x = increment(increment(2));
  cout << "Result: " << x << endl << "Log: " << endl << log.str();
}
\end{asciic}

Fiecare apel al lui \lstinline$increment$ produce un mesaj. Mesajele se acumulează.
\end{frame}

\begin{frame}[fragile]{Stare în C}
\begin{asciic}
#include <iostream>
using namespace std;
int calls;
int increment(int x) {
  calls++;
  return x + calls;
}

int main() {
  int x = increment(increment(2));
  cout << "Result: " << x << endl << "#Calls: " << calls << endl;
}
\end{asciic}

Fiecare apel al lui \lstinline$increment$ citește starea existentă și o modifică.
\end{frame}

\begin{frame}[fragile]{Logging în Haskell}
\begin{block}{Funcția originală}
\vspace{-2ex}
\begin{asciihs}
increment :: Int -> Int
increment x = x + 1
\end{asciihs}
\end{block}
\begin{block}{Funcție cu logging}
„Îmbogățim” la rezultatul funcției cu mesajul de log.
\begin{asciihs}
logIncrement :: Int -> (Int,String)
logIncrement x = (x + 1, "Called increment with argument " ++ show x ++ "\n")
\end{asciihs}
\onslide<2->
\alert{Problemă:} Cum calculăm „\lstinline$logIncrement (logIncrement x)$”?
\end{block}
\end{frame}


\begin{frame}[fragile]{Stare în Haskell}
\begin{block}{Funcția originală în C}
\vspace{-2ex}
\begin{asciic}
int increment(int x) {
  return x + calls++;
}
\end{asciic}
\end{block}
\begin{block}{Funcția cu stare în Haskell}
Rezultatul este acum o funcție, care dată fiind starea dinaintea executiei, produce un rezultat (folosind eventual starea) și starea cea nouă.
\begin{asciihs}
type State = Integer
stateIncrement :: Int -> (State -> (Int, State))
stateIncrement x = f
  where f calls = (x+calls, calls+1)
\end{asciihs}

\alert{Problemă:} Cum calculăm ”\lstinline$stateIncrement (stateIncrement x)$”?
\end{block}
\end{frame}

\begin{frame}[fragile]{Computații nedeterministe}
\begin{block}{Exemplu matematic}
\begin{itemize}
\item Ecuația $y^2 = x$ are două soluții: $\sqrt{x}$ și $-\sqrt{x}$ (complexe, dacă $x<0$).
\item Putem rezolva $z^4 = x$ folosind rezolvarea de mai sus, astfel:
\begin{itemize}
\item Notăm $y = z^2$ și rezolvăm $y^2 = x$ cu soluțiile $\sqrt{x}$ și $-\sqrt{x}$
\item Pentru fiecare soluție $s$ rezolvam $z^2 = s$, obținând $\sqrt{\sqrt{x}}$, $-\sqrt{\sqrt{x}}$, $\sqrt{-\sqrt{x}}$ și $-\sqrt{-\sqrt{x}}$
\end{itemize}
\end{itemize}
\end{block}

\begin{block}{Computație nedeterministă în Haskell}
Rezultatul funcției e listă tuturor valorilor posibile.
\begin{asciihs}
ndSqrt :: Floating a => a -> [a]
ndSqrt x = [sqrt x, -(sqrt x)]
\end{asciihs}
\alert{Problemă:} Cum calculăm ”\lstinline$ndSqrt (ndSqrt x)$”?

\end{block}
\end{frame}

\begin{frame}[fragile]{Cum compunem funcții cu efecte laterale}
\begin{block}{Problema generală}
Dată fiind funcția \lstinline$f :: a -> m b$ și funcția  \lstinline$g :: b -> m c$, vreau să obțin o funcție \lstinline$g # f :: a -> m c$ care este „compunerea” lui $g$ și $f$, propagând efectele laterale.
\end{block}

\begin{block}{Soluție}
Transform \lstinline$g :: b -> m c$ în \lstinline$bind g :: m b -> m c$ pentru un bind „cu proprietăți bune”
\begin{asciihs}
bind :: (b -> m c) -> m b -> m c
(#) :: (b -> m c) -> (a -> m b) -> (a -> m c)
g # f = bind g . f
\end{asciihs}
\end{block}
\end{frame}




\section{Monade}
%
%\begin{frame}[fragile]{Monoizi}
%Un monoid e o structură algebrică dată de o operație binară \lstinline$(@@)$ și o valoare u, astfel încât \lstinline$(@@)$ este asociativă și are element neutru u.
%
%\begin{asciihs}
%   u @@ x              =   x
%   x @@ u              =   x
%   (x @@ y) @@ z       =   x @@ (y @@ z)
%\end{asciihs}
%\begin{block}
%{Exemple de monoizi}
%\begin{itemize}
%\item \lstinline$(+)$ și \lstinline$0$
%\item \lstinline$(*)$ și \lstinline$1$
%\item \lstinline$(||)$ și \lstinline$False$
%\item \lstinline$(&&)$ și \lstinline$True$
%\item \lstinline$(++)$ și \lstinline$[]$
%\item \lstinline$(>>)$ și \lstinline$done$
%\end{itemize}
%\end{block}
%\end{frame}
%
%
%\begin{frame}[fragile]{Monade}
%\lstinline$(>>)$ and \lstinline$done$ verifică axiomele de monoid.
%\begin{asciihs}
%   done >> m          =   m
%   m >> done          =   m
%   (m >> n) >> o      =   m >> (n >> o)
%\end{asciihs}
%
%\vfill
%În mod asemănator, \lstinline$(>>=)$ și \lstinline$return$ verifică axiomele de \structure{monadă}
%\begin{asciihs}
%  return v >>= \x -> m        =  m[v/x]
%  m >>= \x -> return x        =  m
%  (m >>= \x -> n) >>= \y-> o  =  m >>= \x -> (n >>= \y -> o)
%\end{asciihs}
%
%\onslide<2>
%\vfill
%Alternativ, folosind ideea de \structure{continuare}:
%\begin{asciihs}
%  return v >>= k    =  k v
%  m >>= return      =  m
%  (m >>= k) >>= k'  =  m >>= \x -> (k x >>= k')
%\end{asciihs}
%\end{frame}
%
%%
%%
%%Similarly, (>>=) and return satisfy the laws of a monad.
%%   return v >>= \x -> m               =   m[x:=v]
%%   m >>= \x -> return x               =   m
%%   (m >>= \x -> n) >>= \y-> o         =   m >>= \x -> (n >>= \y -> o)
%%\begin{frame}[fragile]{}
%%\begin{asciihs}
%%\end{asciihs}
%%\end{frame}
%
%%Laws of Let
%%We know that (>>) and done satisfy the laws of a monoid.
%%   done >> m           =   m
%%   m >> done           =   m
%%   (m >> n) >> o       =   m >> (n >> o)
%%
%%                    =   let x = m in (let y = n in o)
%\begin{frame}[fragile]{Monade, notație \lstinline$do$ și \lstinline$let$}
%\lstinline$(>>=)$ și \lstinline$return$ verifică axiomele de monadă.
%\begin{asciihs}
%  return v >>= \x -> m        =  m[v/x]
%  m >>= \x -> return x        =  m
%  (m >>= \x -> n) >>= \y-> o  =  m >>= \x -> (n >>= \y -> o)
%\end{asciihs}
%
%%
%%The three monad laws have analogues in “let” notation.
%Axiome analoage pentru let:
%\begin{asciihs}
%   let x = v in m   =   m[v/x]
%   let x = m in x   =   m
%   let y = (let x = m in n) in o
%                    =   let x = m in (let y = n in o)
%\end{asciihs}
%
%\onslide<2>
%Notație \lstinline$do$ pentru expresiile de mai sus:
%\begin{asciihs}
%  do {x <- return v ; m}         =  m[v/x] 
%  do {x <- m ; return x}         =  m
%  do {y <- do {x <- m ; n} ; o}  =  do {x <- m ; 
%                                        do {y <- n ; o}}
%\end{asciihs}
%
%\end{frame}
%
%\begin{frame}[fragile]{Axiomele lui let și efectele laterale}
%\begin{asciihs}
%   let x = v in m   =   m[v/x]
%   let x = m in x   =   m
%   let y = (let x = m in n) in o
%                    =   let x = m in (let y = n in o)
%\end{asciihs}
%Axiomele acestea țin și în limbaje cu efecte laterale dacă v e valoare.
%
%În limbaje „pure” precum Haskell, prima axiomă poate fi întărită astfel:
%\begin{asciihs}
%   let x = n in m   =   m[n/x]
%\end{asciihs}
%cu înțelesul că o variabilă poate fi înlocuită peste tot cu orice termen. Deoarece limbajul este pur, evaluarea lui n nu va produce efecte laterale, indiferent de câte ori este evaluat.
%\end{frame}
%%“Let” in languages with and without effects
%%   let x = v in m   =   m[x:=v]
%%   let x = m in x   =   m
%%   let y = (let x = m in n) in o
%%                    =   let x = m in (let y = n in o)
%%
%%These laws hold even in languages with side effects. For the first law to be true, v
%%must be not an arbitrary term but a value, such as a constant or a variable (but not
%%a function application). A value immediately evaluates to itself, hence it can have
%%no side effects.
%%While in such languages one only has the above three laws for “let”, in Haskell
%%one has a much stronger law, where one may replace a variable by any term, rather
%%than by any value.
%%   let x = n in m         =   m[x:=n]


\begin{frame}[fragile]{Clasa de tipuri \lstinline$Monad$}
\begin{asciihs}
class Applicative m => Monad m where
    (>>=) :: m a -> (a -> m b) -> m b
    return :: a -> m a
    return = pure    
\end{asciihs}
\begin{itemize}
\item m a --- tipul \structure{computațiilor} care produc rezultate de tip a (și au efecte laterale)
\item Tipul \lstinline$a -> m b$ este tipul \structure{continuărilor} / a funcțiilor cu efecte laterale

\item \lstinline$(>>=)$ este operația de „secvențiere” a computațiilor

\item \lstinline$bind k ma = ma >>= k$, adică \lstinline$bind = flip (>>=)$
\item \lstinline$(g # f) x = (bind g . f) x = bind g (f x) = f x >>= g$

\end{itemize}
\end{frame}

\subsection{Instanțe}
\begin{frame}[fragile]{Logging în Haskell}
\begin{block}{Funcție cu logging}
\begin{asciihs}
data Log a = Log { val :: a, log :: String }

logIncrement :: Int -> Log Int
logIncrement x = Log (x + 1) ("Called increment with argument " ++ show x ++ "\n")

logIncrement2 :: Int -> Log Int
logIncrement2 x = logIncrement x >>= logIncrement
\end{asciihs}
\end{block}
\onslide<2>
\begin{block}{Instanța Monad pentru Log}
\begin{asciihs}
instance Monad Log where
  return a = Log a ""
  ma >>= k = Log {val = val mb, log = log ma ++ log mb}
    where mb = k (val ma)
\end{asciihs}
\end{block}
\end{frame}


\begin{frame}[fragile]{Stare în Haskell}
\begin{block}{Funcția cu stare în Haskell}
\vspace{-2ex}
\begin{asciihs}
data State state val = State {apply :: state -> (val,state)}

stateIncrement :: Int -> State Int Int
stateIncrement x = State f
  where f calls = (x+calls, calls+1)

stateIncrement2 :: Int -> State Int Int
stateIncrement2 x = stateIncrement x >>= stateIncrement
\end{asciihs}
\end{block}
\onslide<2>
\begin{block}{Instanța Monad pentru stare}
\vspace{-2ex}
\begin{asciihs}
instance Monad (State state) where
  return a = State (\ s -> (a, s)) 
  ma >>= k = State g
    where g state = let (val, state') = apply ma state 
                     in apply (k val) state'
\end{asciihs}
\end{block}
\end{frame}

\begin{frame}[fragile]{Computații nedeterministe}
\begin{block}{Computație nedeterministă în Haskell}
Rezultatul funcției e listă tuturor valorilor posibile.
\begin{asciihs}
ndSqrt :: Floating a => a -> [a]
ndSqrt x = [sqrt x, -(sqrt x)]

ndSqrt2 :: Floating a => a -> [a]
ndSqrt2 x = ndSqrt x >>= ndSqrt
\end{asciihs}
\end{block}

\onslide<2>
\begin{block}{Instanța Monad pentru liste}
\vspace{-2ex}
\begin{asciihs}
instance Monad [] where
  return a = [a] 
  xs >>= k = [y | x <- xs, y <- k x]
\end{asciihs}
\end{block}
\end{frame}


\begin{frame}[fragile]{Proprietățile monadelor}
\begin{block}{Pe scurt}
Operația de compunere \lstinline$#$ a continuărilor este asociativă și are element neutru \lstinline$return$
\end{block}

\begin{block}{Pe mai putin scurt}
\begin{description}
\item[NeutruD] \lstinline$(return x) >>= g = g x$
\item[NeutruS] \lstinline$x >>= return = x$
\item[Assoc] \lstinline$(fm >>= g) >>= h = fm >>= \ x -> (g x >>= h)$
\end{description}
\end{block}

\end{frame}


%The Monad type class




%        Part VII
%
%Roll your own monad—IO

\section{Să ne definim propria monadă IO}


\begin{frame}[fragile]{Monada MyIO}{partea I}
\begin{asciihs}
module MyIO(MyIO, myPutChar, myGetChar, convert) where

type Input = String
type Output = String

data MyIO a = MyIO { apply :: Input -> (a, Input, Output) }

\end{asciihs}

\begin{block}{Observație: Tipul MyIO is abstract}
\begin{itemize} 
\item Sunt exportate doar tipul MyIO, myPutChar, myGetChar, convert 

(și operațiile de monadă)
\item Nu este exportat constructorul MyIO și nici operația apply
\end{itemize}
\end{block} 
\end{frame}

%My own IO monad (1)
%


\begin{frame}[fragile]{Monada MyIO}{partea II}
\begin{asciihs}
 myPutChar :: Char -> MyIO ()
 myPutChar c = MyIO (\ input -> ((), input, [c]))

 myGetChar :: MyIO Char
 myGetChar = MyIO (\ (ch:input') -> (ch, input', ""))
\end{asciihs}

\begin{block}
{Exemplu}
\begin{asciihs}
   apply myGetChar "abc" == ('a',    "bc", "")
   apply myGetChar "bc"   == ('b',   "c", "")
   apply (myPutChar 'A') "def" ==    ((), "def", "A")
   apply (myPutChar 'B') "def" ==    ((), "def", "B")
\end{asciihs}
\end{block}

\end{frame}



\begin{frame}[fragile]{Monada MyIO}{partea III}
\begin{asciihs}
   instance Monad MyIO where
     return x = MyIO (\ input -> (x, input, ""))
     m >>= k  = MyIO (\ input ->
        let (x, input1, output1) = apply m input
            (y, input2, output2) = apply (k x) input1 
         in (y, input2, output1 ++ output2))
\end{asciihs}

\begin{block}
{Exemplu}
\vspace{-2ex}
\begin{asciihs}
   apply
     (myGetChar >>= \x -> myGetChar >>= \y -> return [x,y])
     "abc"
   == ("ab", "c", "")
   apply (myPutChar 'A' >> myPutChar 'B') "def"
   == ((), "def", "AB")
   apply (myGetChar >>= \x -> myPutChar (toUpper x)) "abc"
   == ((), "bc", "A")
\end{asciihs}
\end{block}

\end{frame}


%My own IO monad (3)
%
%For example

\begin{frame}[fragile]{Monada MyIO}{partea IV}
\begin{asciihs}
   convert :: MyIO () -> IO ()
   convert m = interact (\ input ->
                   let (x, input', output) = apply m input
                    in output)
\end{asciihs}

Unde
\begin{asciihs}
   interact :: (String -> String) -> IO ()
\end{asciihs}
face parte din biblioteca standard, si face următoarele:
\begin{itemize}
\item Citește stream-ul de intrare la un șir de caractere (leneș)
\item Aplică funcția dată ca parametru acestui șir
\item Trimite șirul rezultat către stream-ul de ieșire (tot leneș)
\end{itemize}
\end{frame}


%My own IO monad (4)
%   convert :: MyIO () -> IO ()
%   convert m = interact (\inp ->
%                   let (x, rem, out) = apply m inp in
%                   out)
%
%Here
%   interact :: (String -> String) -> IO ()
%
%is part of the standard prelude. The entire input is converted to a string (lazily) and
%passed to the function, and the result from the function is printed as output (also
%lazily).


\begin{frame}[fragile]{Folosirea monadei MyIO}
{partea I}
\begin{asciihs}
  module MyEcho where

  import Char
  import MyIO

  myPutStr :: String -> MyIO ()
  myPutStr = foldr (>>) (return ()) . map myPutChar

  myPutStrLn :: String -> MyIO ()
  myPutStrLn s = myPutStr s >> myPutChar '\n'
\end{asciihs}
\end{frame}


%Using my own IO monad (1)

\begin{frame}[fragile]{Folosirea monadei MyIO}
{partea II}
\vspace{-2ex}
\begin{asciihs}
  myGetLine :: MyIO String
  myGetLine = myGetChar >>= \x ->
               if x == '\n' then
                 return []
               else
                 myGetLine >>= \xs ->
                 return (x:xs)

  myEcho :: MyIO ()
  myEcho = myGetLine >>= \line ->
            if line == "" then
              return ()
            else
              myPutStrLn (map toUpper line) >>
              myEcho

  main :: IO ()
  main = convert myEcho
\end{asciihs}
\end{frame}


%Using my own IO monad (2)

\begin{frame}[fragile]{În execuție}
{partea I}
\begin{asciihs}
  10-monade$ runghc MyEcho
  This is a test.
  THIS IS A TEST.
  It is only a test.
  IT IS ONLY A TEST.
  Were this a real emergency, you'd be dead now.
  WERE THIS A REAL EMERGENCY, YOU'D BE DEAD NOW.

  10-monade$
\end{asciihs}
\end{frame}

\begin{frame}[fragile]{Folosind notația \lstinline$do$}
\vspace{-2ex}
\begin{asciihs}
  myGetLine :: MyIO String
  myGetLine = do {
                 x <- myGetChar;
                 if x == '\n' then
                   return []
                 else do {
                   xs <- myGetLine;
                   return (x:xs)
                 }
               }
  myEcho :: MyIO ()
  myEcho = do {
              line <- myGetLine;
              if line == "" then
                return ()
              else do {
                myPutStrLn (map toUpper line);
                myEcho
              } }
\end{asciihs}
\end{frame}


%You can use “do” notation, too
%     Part VIII
%
%The monad of lists

\end{document}



